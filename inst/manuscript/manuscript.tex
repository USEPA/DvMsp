\documentclass[]{elsarticle} %review=doublespace preprint=single 5p=2 column
%%% Begin My package additions %%%%%%%%%%%%%%%%%%%
\usepackage[hyphens]{url}

  \journal{Methods in Ecology and Evolution} % Sets Journal name


\usepackage{lineno} % add
  \linenumbers % turns line numbering on
\providecommand{\tightlist}{%
  \setlength{\itemsep}{0pt}\setlength{\parskip}{0pt}}

\usepackage{graphicx}
\usepackage{booktabs} % book-quality tables
%%%%%%%%%%%%%%%% end my additions to header

\usepackage[T1]{fontenc}
\usepackage{lmodern}
\usepackage{amssymb,amsmath}
\usepackage{ifxetex,ifluatex}
\usepackage{fixltx2e} % provides \textsubscript
% use upquote if available, for straight quotes in verbatim environments
\IfFileExists{upquote.sty}{\usepackage{upquote}}{}
\ifnum 0\ifxetex 1\fi\ifluatex 1\fi=0 % if pdftex
  \usepackage[utf8]{inputenc}
\else % if luatex or xelatex
  \usepackage{fontspec}
  \ifxetex
    \usepackage{xltxtra,xunicode}
  \fi
  \defaultfontfeatures{Mapping=tex-text,Scale=MatchLowercase}
  \newcommand{\euro}{€}
\fi
% use microtype if available
\IfFileExists{microtype.sty}{\usepackage{microtype}}{}
\bibliographystyle{elsarticle-harv}
\ifxetex
  \usepackage[setpagesize=false, % page size defined by xetex
              unicode=false, % unicode breaks when used with xetex
              xetex]{hyperref}
\else
  \usepackage[unicode=true]{hyperref}
\fi
\hypersetup{breaklinks=true,
            bookmarks=true,
            pdfauthor={},
            pdftitle={A comparison of design-based and model-based approaches for finite population spatial data.},
            colorlinks=false,
            urlcolor=blue,
            linkcolor=magenta,
            pdfborder={0 0 0}}
\urlstyle{same}  % don't use monospace font for urls

\setcounter{secnumdepth}{5}
% Pandoc toggle for numbering sections (defaults to be off)

% Pandoc citation processing

% Pandoc header

\usepackage{bm} \usepackage{bbm} \usepackage{color} \DeclareMathOperator{\var}{{var}} \DeclareMathOperator{\cov}{{cov}} \usepackage{caption} \usepackage{subcaption} \usepackage{setspace} \doublespacing

\begin{document}
\begin{frontmatter}

  \title{A comparison of design-based and model-based approaches for finite
population spatial data.}
    \author[USEPA]{Michael Dumelle\corref{1}}
  
    \author[STLAW]{Matt Higham}
  
    \author[NOAA]{Jay M. Ver Hoef}
  
    \author[USEPA]{Anthony R. Olsen}
  
    \author[OSU]{Lisa Madsen}
  
      \address[USEPA]{United States Environmental Protection Agency, 200 SW 35th St,
Corvallis, Oregon, 97333}
    \address[STLAW]{Saint Lawrence University Department of Mathematics, Computer Science,
and Statistics, 23 Romoda Drive, Canton, New York, 13617}
    \address[NOAA]{Marine Mammal Laboratory, Alaska Fisheries Science Center, National
Oceanic and Atmospheric Administration, Seattle, Washington, 98115}
    \address[OSU]{Oregon State University Department of Statistics, 239 Weniger Hall,
Corvallis, Oregon, 97331}
      \cortext[1]{Corresponding Author: Michael Dumelle (Dumelle.Michael@epa.gov)}
  
  \begin{abstract}
  \begin{enumerate}
  \def\labelenumi{\arabic{enumi}.}
  \tightlist
  \item
    The design-based and model-based approaches to frequentist statistical
    inference lie on fundamentally different foundations. In the
    design-based approach, inference depends on random sampling. In the
    model-based approach, inference depends on distributional assumptions.
    We compare the approaches for finite population spatial data.
  \item
    We provide relevant background for the design-based and model-based
    approaches and then study their performance using simulations and an
    analysis of real mercury concentration data. In the simulations, a
    variety of sample sizes, location layouts, dependence structures, and
    response types are considered. In the simualations and real data
    analysis, the population mean is the parameter of interest and
    performance is measured using statistics like bias, squared error, and
    interval coverage.
  \item
    When studying the simulations and mercury concentration data, we found
    that regardless of the strength of spatial dependence in the data,
    sampling plans that incorporate spatial locations (spatially balanced
    samples) generally outperform sampling plans that ignore spatial
    locations (non-spatially balanced samples). We also found that
    model-based approaches tend to outperform design-based approaches,
    even when the data are skewed (and by consequence, the model-based
    distributional assumptions violated). The performance gap between
    these approaches is small when spatially balanced samples are used but
    large when non-spatially balanced samples are used. This suggests that
    the sampling choice (whether to select a sample that is spatially
    balanced) is most important when performing design-based inference.
  \item
    There are many benefits and drawbacks to the design-based and
    model-based approaches for finite population spatial data that
    practitioners must consider when choosing between them. We provide
    relevant background contextualizing each approach and study their
    properties in a variety of scenarios, making recommendations for use
    based on the practitioner's goals.
  \end{enumerate}
  \end{abstract}
  
 \end{frontmatter}

\hypertarget{keywords}{%
\section*{Keywords}\label{keywords}}
\addcontentsline{toc}{section}{Keywords}

Design-based inference; Finite Population Block Kriging (FPBK);
Generalized Random Tessellation Stratified (GRTS) algorithm; Model-based
inference; Spatially balanced sampling; Spatial covariance

\hypertarget{sec:introduction}{%
\section{Introduction}\label{sec:introduction}}

There are two general approaches for using data to make frequentist
statistical inferences about a population: design-based and model-based.
When data cannot be collected for all units in a population (i.e.,
population units), data are collected on a subset of the population
units. This subset of population units is called a sample. In the
design-based approach, inferences about the underlying population are
informed via a probabilistic process that randomly assigns some
population units to be in the sample. Alternatively, in the model-based
approach, inferences are made from specific assumptions about the
underlying process generating the data. Each paradigm has a deep
historical context (Sterba, 2009) and its own set of benefits and
drawbacks (Hansen et al., 1983).

Though the design-based and model-based approaches apply to statistical
inference in a broad sense, we focus on comparing these approaches for
spatial data. We define spatial data as data that incorporates the
specific locations of the population units into either the sampling or
estimation process. De Gruijter and Ter Braak (1990) give an early
comparison of design-based and model-based approaches for spatial data,
quashing the belief that design-based approaches could not be used for
spatially correlated data. Since then, there have been several general
comparisons between design-based and model-based approaches for spatial
data (Brus and De Gruijter, 1997; Brus, 2021; Ver Hoef, 2002, 2008; Wang
et al., 2012). Cooper (2006) reviews the two approaches in an ecological
context before introducing a ``model-assisted'' variance estimator that
combines aspects from each approach. In addition to Cooper (2006), there
has been substantial research and development into estimators that use
both design and model-based principles (see e.g., Sterba (2009) and
Cicchitelli and Montanari (2012), and see Chan-Golston et al. (2020) for
a Bayesian approach).

Certainly comparisons between design-based and model-based approaches to
spatial data have been studied. But no numerical comparison has been
made between design-based approaches that incorporate spatial
information and model-based approaches. In this manuscript, we compare
design-based approaches that incorporate spatial information to
model-based approaches for finite population spatial data. A finite
population contains a finite number of population units (we assume the
finite number is known); an example is lakes (treated as a whole with
the lake centroid representing location) in the contiguous United
States. Though we focus on finite populations, these comparisons
generalize to infinite populations as well. An infinite population
contains an infinite number of population units; an example is locations
within a single lake.

The rest of the manuscript is organized as follows. In Section
\ref{sec:background}, we introduce and provide relevant background for
the design-based and model-based approaches to finite population spatial
data. In Section \ref{sec:mm}, we describe how we compare performance of
the approaches with a simulation study and an analysis of real data that
contains mercury concentration in lakes located in the contiguous United
States. In Section \ref{sec:results}, we present results from the
simulation study and the mercury concentration analysis. And in Section
\ref{sec:discussion}, we end with a discussion and provide directions
for future research.

\hypertarget{sec:background}{%
\subsection{Background}\label{sec:background}}

The design-based and model-based approaches incorporate randomness in
fundamentally different ways. In this section, we describe the role of
randomness for each approach and the subsequent effects on statistical
inferences for spatial data.

\hypertarget{subsec:dvm_compare}{%
\subsubsection{Comparing Design-Based and Model-Based
Approaches}\label{subsec:dvm_compare}}

The design-based approach assumes the population is fixed. Randomness is
incorporated via the selection of population units according to a
sampling design. A sampling design assigns a positive probability of
inclusion (inclusion probability) in the sample to each population unit.
These inclusion probabilities are later used to analyze data. Some
examples of commonly used sampling designs include simple random
sampling, stratified random sampling, and cluster sampling.

When sampling designs incorporate spatial locations into sampling, we
call the resulting samples ``spatially balanced.'' One approach to
selecting spatially balanced samples is the Generalized Random
Tessellation Stratified (GRTS) algorithm (Stevens and Olsen, 2004),
which we discuss in more detail in Section \ref{subsec:spb_design}. When
sampling designs do not incorporate spatial locations into sampling, we
call the resulting samples ``non-spatially balanced.''

Fundamentally, the design-based approach combines the randomness of the
sampling design with the data collected via the sample to justify the
estimation and uncertainty quantification of fixed, unknown parameters
of a population (e.g., a population mean). Treating the data as fixed
and incorporating randomness through the sampling design yields
estimators having very few other assumptions. Confidence intervals for
these types of estimators are typically derived using limiting arguments
that incorporate all possible samples. Sample means, for example, are
asymptotically normal (Gaussian) by the Central Limit Theorem (under
some assumptions). If we repeatedly select samples from the population,
then 95\% of all 95\% confidence intervals constructed from a procedure
with appropriate coverage will contain the true, fixed mean. Särndal et
al. (2003) and Lohr (2009) provide thorough reviews of the design-based
approach.

The model-based approach assumes the data are a random realization of a
data-generating stochastic process. Randomness is incorporated through
distributional assumptions on this process. Strictly speaking,
randomness need not be incorporated through random sampling, though
Diggle et al. (2010) warn against preferential sampling. Preferential
sampling occurs when the process generating the data locations and the
process being modeled are not independent of one another. To guard
against preferential sampling, model-based approaches often still
implement some form of random sampling. When model-based approaches
implement random sampling, the inclusion probabilities are ignored when
analyzing the data (in contrast to the design-based approach, which
relies on these inclusion probabilities to analyze the data).

Instead of estimating fixed, unknown population parameters, as in the
design-based approach, often the goal of model-based inference is to
predict a realized variable, or value. For example, suppose the realized
mean of all population units is the value of interest. Instead of
\emph{estimating} a fixed, unknown mean, we are \emph{predicting} the
value of the mean, a random variable. Prediction intervals are then
derived using assumptions of the data-generating stochastic process. If
we repeatedly generate response values from the same data-generating
stochastic process and select samples, then 95\% of all 95\% prediction
intervals constructed from a procedure with appropriate coverage will
contain their respective realized means. Cressie (1993) and
Schabenberger and Gotway (2017) provide thorough reviews of model-based
approaches for spatial data. In Fig. \ref{fig:fig1}, we provide a visual
comparison of the design-based and model-based approaches (Ver Hoef
(2002) and Brus (2021) provide similar figures).

\begin{figure}
  \centering
  \includegraphics[width = 1\linewidth]{figures/dvm_comp.jpeg}
  \caption{A visual comparison of the design-based and model-based approaches. In the top row, the design-based approach is highlighted. There is one fixed population with nine population units and three random samples of size four (points circled are those sampled). The response values at each site are fixed, but we obtain different estimates for the mean response in each random sample. In the bottom row, the model-based approach is highlighted. There are three realizations of the same data-generating stochastic process that are all sampled at the same four locations. The data-generating stochastic process has a single mean, but the mean of the nine population units is different in each of the three realizations.}
  \label{fig:fig1}
\end{figure}

\hypertarget{subsec:spb_design}{%
\subsubsection{Spatially Balanced Design and
Analysis}\label{subsec:spb_design}}

We previously mentioned that the design-based approach can be used to
select spatially balanced samples (samples that incorporate spatial
locations of the population units). Spatially balanced samples are
useful because parameter estimates from these samples tend to vary less
than parameter estimates from samples that are not spatially balanced
(Barabesi and Franceschi, 2011; Benedetti et al., 2017; Grafström and
Lundström, 2013; Robertson et al., 2013; Stevens and Olsen, 2004; Wang
et al., 2013). The first spatially balanced sampling algorithm to see
widespread use was the Generalized Random Tessellation Stratified (GRTS)
algorithm (Stevens and Olsen, 2004). To quantify the spatial balance of
a sample, Stevens and Olsen (2004) proposed loss metrics based on
Voronoi polygons (Dirichlet Tessellations). After the GRTS algorithm was
developed, several other spatially balanced sampling algorithms emerged,
including the Local Pivotal Method (Grafström et al., 2012; Grafström
and Matei, 2018), Spatially Correlated Poisson Sampling (Grafström,
2012), Balanced Acceptance Sampling (Robertson et al., 2013),
Within-Sample-Distance Sampling (Benedetti and Piersimoni, 2017), and
Halton Iterative Partitioning Sampling (Robertson et al., 2018). In this
manuscript, we select spatially balanced samples using the Generalized
Random Tessellation Stratified (GRTS) algorithm because it has several
attractive properties: the GRTS algorithm accommodates finite and
infinite sampling frames, equal, unequal, and proportional (to size)
inclusion probabilities, legacy (historical) sampling (Foster et al.,
2017), a minimum distance between units in a sample, and replacement
units (replacement units are population units that can be sampled when a
population unit originally selected can no longer be sampled). The GRTS
algorithm selects samples by utilizing a particular mapping between
two-dimensional and one-dimensional space that preserves proximity
relationships. Via this mapping, units in two-dimensional space are
partitioned using a hierarchical address. This hierarchical address is
used to map population units to a one-dimensional line. On the one
dimensional line, each population unit's line length equals its
inclusion probability. Then, a systematic sample of population units is
selected on the line and mapped back to two-dimensional space, yielding
the desired sample. Stevens and Olsen (2004) provide more technical
details.

After selecting a sample and collecting data, unbiased estimates of
population means and totals can be obtained using the Horvitz-Thompson
estimator (Horvitz and Thompson, 1952). If \(\tau\) is a population
total, the Horvitz-Thompson estimator for \(\tau\), denoted by
\(\hat{\tau}_{ht}\), is is given by \begin{align}\label{eq:ht}
  \hat{\tau}_{ht} = \sum_{i = 1}^n Z_i \pi_i^{-1},
\end{align} where \(Z_i\) is the value of the \(i\)th population unit in
the sample, \(\pi_i\) is the inclusion probability of the \(i\)th
population unit in the sample, and \(n\) is the sample size. An estimate
of the population mean is obtained by dividing \(\hat{\tau}_{ht}\) by
\(N\), the number of population units.

It is also important to quantify the uncertainty in \(\hat{\tau}_{ht}\).
Horvitz and Thompson (1952) and Sen (1953) provide variance estimators
for \(\hat{\tau}_{ht}\), but these estimators have two drawbacks. First,
they rely on calculating \(\pi_{ij}\), the probability that population
unit \(i\) and population unit \(j\) are both in the sample -- this
quantity can be challenging if not impossible to calculate analytically.
Second, these estimators ignore the spatial locations of the population
units. To address these two drawbacks simultaneously, Stevens and Olsen
(2003) proposed the local neighborhood variance estimator. The local
neighborhood variance estimator does not rely on \(\pi_{ij}\) and
incorporates spatial locations -- for technical details see Stevens and
Olsen (2003). Stevens and Olsen (2003) show the local neighborhood
variance estimator tends to reduce the estimated variance of
\(\hat{\tau}\) and yield narrower confidence intervals compared to
variance estimators that ignore spatial locations.

\hypertarget{finite-population-block-kriging}{%
\subsubsection{Finite Population Block
Kriging}\label{finite-population-block-kriging}}

Finite Population Block Kriging (FPBK) is a model-based approach that
expands the geostatistical Kriging framework to the finite population
setting (Ver Hoef, 2008). Instead of developing inference based on a
specific sampling design, we assume the data are generated by a spatial
stochastic process. We summarize some of the basic principles of FBPK
next -- for technical details, see Ver Hoef (2008). Let
\({\mathbf{z} \equiv \{\text{z}(s_1), \text{z}(s_2), . . . , \text{z}(s_N) \}}\)
be an \(N \times 1\) response vector at locations \(s_1\), \(s_2\), . .
. , \(s_N\) that can be measured at the \(N\) population units. Suppose
we want to use a sample to predict some linear function of the response
variable, \(f(\mathbf{z}) = \mathbf{b}^\prime \mathbf{z}\), where
\(\mathbf{b}^\prime\) is a \(1 \times N\) vector of weights (e.g, the
population mean is represented by a weights vector whose elements all
equal one). Denoting quantities that are part of the sampled population
units with a subscript \emph{s} and quantities that are part of the
unsampled population units with a subscript \emph{u}, let

\begin{equation}
\begin{pmatrix} \label{equation:Zmarginal}
    \mathbf{z}_s      \\
    \mathbf{z}_u
\end{pmatrix}
=
\begin{pmatrix}
  \mathbf{X}_s    \\
  \mathbf{X}_u
\end{pmatrix}
\bm{\beta} +
\begin{pmatrix}
\bm{\delta}_s    \\
\bm{\delta}_u
\end{pmatrix},
\end{equation} where \(\mathbf{X}_s\) and \(\mathbf{X}_u\) are the
design matrices for the sampled and unsampled population units,
respectively, \(\bm{\beta}\) is the parameter vector of fixed effects,
and \(\bm{\delta} \equiv [\bm{\delta}_s \,\, \bm{\delta}_u]'\), where
\(\bm{\delta}_s\) and \(\bm{\delta}_u\) are random errors for the
sampled and unsampled population units, respectively.

FBPK assumes \(\bm{\delta}\) in Equation\(~\)\ref{equation:Zmarginal}
has mean-zero and a spatial dependence structure that can be modeled
using a covariance function. This covariance function is commonly
assumed to be non-negative, second-order stationary (depending only on
the distance between population units), isotropic (independent of
direction), and decay with distance between population units (Cressie,
1993). Henceforth, it is implied that we have made these same
assumptions regarding \(\bm{\delta}\), though Chiles and Delfiner
(1999), pp.~80-93 discuss covariance functions that are not second-order
stationary, not isotropic, or not either. A variety of flexible
covariance functions can be used to model \(\bm{\delta}\) (Cressie,
1993); one example is the exponential covariance function (Cressie
(1993) provides a thorough list of spatial covariance functions). The
\(i,j\)th element of the exponential covariance matrix,
\(\mathop{\mathrm{{cov}}}(\bm{\delta})\), is \mbox{}
\begin{align}\label{equation:expcov}
\mathop{\mathrm{{cov}}}(\delta_i, \delta_j) = 
\begin{cases} 
\sigma^2_{1}\exp(-h_{i,j}/\phi) & h_{i,j} > 0 \\
\sigma^2_{1} + \sigma^2_2 & h_{i,j} = 0
\end{cases}
,
\end{align} where \(\sigma^2_{1}\) is the variance parameter quantifying
the variability that is dependent (coarse-scale), \(\sigma^2_{2}\) is
the variance parameter quantifying the variability that is independent
(fine-scale), \(\phi\) is the range parameter measuring the
distance-decay rate of the covariance, and \(h_{i,j}\) is the Euclidean
distance between population units \(i\) and \(j\). The proportion of
variability attributable to dependent random error is
\(\sigma^2_{1} / (\sigma^2_{1} + \sigma^2_{2})\). Similarly, the
proportion of variability attributable to independent random error is
\(\sigma^2_{2} / (\sigma^2_{1} + \sigma^2_{2})\). Finally we note that
\(\sigma^2_{1}\) and \(\sigma^2_{2}\) are often called the partial sill
and nugget, respectively.

With the above model formulation, the Best Linear Unbiased Predictor
(BLUP) for \(f(\mathbf{b}'\mathbf{z})\) and its prediction variance can
be computed. While details of the derivation are in Ver Hoef (2008), we
note here that the predictor and its variance are both moment-based,
meaning that they do not rely on any distributional assumptions.

Other approaches, such as k-nearest-neighbors (Fix and Hodges, 1989; Ver
Hoef and Temesgen, 2013) and random forest (Breiman, 2001), among
others, could also be used to obtain predictions for a mean or total
from finite population spatial data. Compared to the k-nearest-neighbors
and random forest approach, we prefer FBPK because it is model-based and
relies on theoretically-based variance estimators leveraging the model's
spatial covariance structure, whereas k-nearest-neighbors and random
forests use ad-hoc variance estimators (Ver Hoef and Temesgen, 2013).
Additionally, Ver Hoef and Temesgen (2013) studied compared FBPK,
k-nearest-neighbors, and random forest in a variety of spatial data
contexts, and FBPK tended to perform best.

\hypertarget{sec:mm}{%
\section{Materials and Methods}\label{sec:mm}}

\hypertarget{sec:mm_sim}{%
\subsection{Simulation Study}\label{sec:mm_sim}}

We used a simulation study to investigate performance of four
sampling-analysis combinations. The first sampling-analysis combination
is IRS-Design. In IRS-Design, samples are selected using the Independent
Random Sampling (IRS) algorithm. The IRS algorithm ignores the spatial
locations of the population units, which implies IRS samples are not
spatially balanced. In IRS-Design, samples are analyzed using the
design-based approach with an IRS variance estimator that does not
incorporate the spatial locations of the units in the sample. The second
sampling-analysis combination is IRS-Model, where samples are selected
using the IRS algorithm and analyzed using the model-based approach via
Restricted Maximum Likelihood (REML) estimation (Harville, 1977;
Patterson and Thompson, 1971; Wolfinger et al., 1994). The third
sampling-analysis combination is GRTS-Design, where samples are selected
using the GRTS algorithm and analyzed using the design-based approach
with the local neighborhood variance estimator. The fourth and final
sampling-analysis combination is GRTS-Model, where samples are selected
using the GRTS algorithm and analyzed using the model-based approach via
REML estimation. These sampling-analysis combinations are also provided
in Table \ref{tab:designanalysis}. Lastly we note that for both the IRS
and GRTS samples, equal inclusion probabilities were assumed for all
population units. When IRS assumes equal inclusion probabilities for all
population units, the algorithm is equivalent to ``simple random
sampling.''

\begin{table}[ht]
\centering
\begin{tabular}{r|ll}
  \hline
 & Design & Model \\ 
  \hline
IRS & IRS-Design & IRS-Model \\ 
  GRTS & GRTS-Design & GRTS-Model \\ 
   \hline
\end{tabular}
\caption{\label{tab:designanalysis} Sampling-analysis combinations in the simulation study. The rows give the two types of sampling designs and the columns give the two types of analyses.} 
\end{table}

Performance for the four sampling-analysis combinations was evaluated in
36 different simulation scenarios. The 36 scenarios resulted from the
crossing of three sample sizes, two location layouts (of the population
units), two response types, and three proportions of dependent random
error. The three sample sizes (\(n\)) were \(n = 50, n = 100,\) and
\(n = 200\). Samples were always selected from a population size (\(N\))
of \(N = 900\). The two location layouts were random and gridded.
Locations in the random layout were randomly generated inside the unit
square (\([0, 1] \times [0, 1]\)). Locations in the gridded layout were
placed on a fixed, equally spaced grid inside the unit square. The two
response types were normal and lognormal. For the normal response type,
the response was simulated using mean-zero random errors with the
exponential covariance (Equation\(~\)\ref{equation:expcov}) for varying
proportions of dependent random error. The proportion of dependent
random error is represented by
\(\sigma^2_1 / (\sigma^2_1 + \sigma^2_2)\), where \(\sigma^2_1\) and
\(\sigma^2_2\) are the dependent random error variance (partial sill)
and independent random error variance (nugget), respectively, from
Equation\(~\)\ref{equation:expcov}. The total variance,
\(\sigma^2_1 + \sigma^2_2\), was always 2. The range was always
\(\sqrt{2} / 3\), which means that the correlation in the dependent
random error decayed to nearly zero at the largest possible distance
between two population units in the domain. For the lognormal response
type, the response was first simulated using the same approach as for
the normal response type, except that the total variance was 0.6931
instead of 2. The response was then exponentiated, yielding a lognormal
random variable whose total variance was 2. The lognormal responses were
used to evaluate performance of the sampling-analysis approaches for
data that were skewed (i.e., not normal).

\begin{table}[ht]
\centering
\begin{tabular}{r|lll}
   \hline
Sample Size (n) & 50 & 100 & 200 \\ 
  Location Layout & Random & Gridded & - \\ 
  Proportion of Dependent Error & 0 & 0.5 & 0.9 \\ 
  Response Type & Normal & Lognormal & - \\ 
   \hline
\end{tabular}
\caption{\label{tab:parmtab} Simulation scenario options. All combinations of sample size, location layout, response type, and proportion of dependent random error composed the 36 simulation scenarios. In each simualtion scenario, the total variance was 2.} 
\end{table}

In each of the 36 simulation scenarios, there were 2000 independent
simulation trials. In each trial, IRS and GRTS samples were selected and
then design-based and model-based analyses were used to estimate
(design-based) or predict (model-based) the mean and construct 95\%
confidence (design-based) or 95\% prediction (model-based) intervals.
Then we recorded the bias, squared error, and interval coverage for all
sampling-analysis combinations. After all 2000 trials, we summarized the
long-run performance of the combinations by calculating average bias,
rMS(P)E (root-mean-squared error for the design-based approaches and
root-mean-squared-prediction error for the model-based approaches), and
the proportion of times the true mean is contained in its 95\%
confidence (design-based) or 95\% prediction (model-based) interval. The
95\% confidence intervals (design-based) and 95\% prediction intervals
(model-based) were constructed using the normal distribution.
Justification for this comes from the asymptotic normality of means via
the Central Limit Theorem (under some assumptions). The IRS algorithm,
IRS variance estimator, GRTS algorithm, and local neighborhood variance
estimator are available in the \texttt{spsurvey} \textbf{\textsf{R}}
package (Dumelle et al., 2021). FPBK is available in the
\texttt{sptotal} \textbf{\textsf{R}} package (Higham et al., 2021).

\hypertarget{sec:mm_app}{%
\subsection{Application}\label{sec:mm_app}}

The United States Environmental Protection Agency (USEPA), states, and
tribes periodically conduct National Aquatic Research Surveys (NARS) to
assess the water quality of various bodies of water in the contiguous
United States. One component of NARS is the National Lakes Assessment
(NLA), which measures various aspects of lake health and water quality
(USEPA, 2012). We will analyze mercury concentration data collected at
986 lakes as part of the 2012 NLA. Although we can calculate the true
mean mercury concentration values for these 986 lakes, here we will
explore whether or not we can obtain an adequately precise estimate for
the realized mean mercury concentration if we sample only 100 of the 986
lakes. For each of the four familiar sampling-analysis combinations
(IRS-Design, IRS-Model, GRTS-Design, and GRTS-Model), we estimate
(design-based) or predict (model-based) the mean mercury concentration
and construct 95\% confidence (design-based) or 95\% prediction
(model-based) intervals from this sample of 100 lakes, which we compare
to the true mean mercury concentration from all 986 lakes.

\hypertarget{sec:results}{%
\section{Results}\label{sec:results}}

\hypertarget{sec:r_sim}{%
\subsection{Simulation Study}\label{sec:r_sim}}

The average bias was nearly zero for all four sampling-analysis
combinations in all 36 scenarios, so we omit a more detailed summary of
those results here. Tables for average bias in all 36 simulation
scenarios are provided in the supporting information.

Fig. \ref{fig:rmspe_eff} shows the relative rMS(P)E of the four sampling
analysis combinations using the random location layout with
``IRS-Design'' as the baseline. The relative rMS(P)E is defined as
\begin{equation*}
\frac{\text{rMS(P)E of sampling-analysis combination}}{\text{rMS(P)E of IRS-Design}},
\end{equation*} When there is no spatial covariance (Fig.
\ref{fig:rmspe_eff}, ``Prop DE: 0'' row), the four sampling-analysis
combinations have approximately equal rMS(P)E. So using the GRTS
algorithm or a model-based analysis does not result in much, if any,
loss in efficiency compared to IRS-Design when there is no spatial
covariance. When there is spatial covariance (Fig. \ref{fig:rmspe_eff},
``Prop DE: 0.5'' and ``Prop DE: 0.9'' rows), GRTS-Model tends to perform
best, followed by GRTS-Design, IRS-Model, and finally IRS-Design, though
the difference in relative rMS(P)E among GRTS-Model, GRTS-Design, and
IRS-Model is relatively small. As the strength of spatial covariance
increases, the gap in rMS(P)E between IRS-Design and the other
sampling-analysis combinations widens. Finally we note that when there
is spatial covariance, IRS-Model outperforms IRS-Design by a large
margin, suggesting that the poor design properties of IRS are largely
mitigated by the model-based analysis. These conclusions are similar to
those observed in the grid location layout, so we omit a grid location
layout figure here. Tables for rMS(P)E in all 36 simulation scenarios
are provided in the supporting information.

\begin{figure}
  \centering
  \includegraphics[width = 1\linewidth]{figures/rmspe_eff.jpeg}
  \caption{Relative rMS(P)E in the simulation study for the four sampling-analysis combinations. The rows indicate the proportion of dependent error and the columns indicate the response type.}
  \label{fig:rmspe_eff}
\end{figure}

Fig. \ref{fig:mse_eff} shows the relative mean standard errors (MStdE)
of the four sampling-analysis combinations using the random location
layout with ``IRS-Design'' as the baseline. The MStdE is defined as
\begin{equation*}
\frac{\text{MStdE of sampling-analysis combination}}{\text{MStdE of IRS-Design}},
\end{equation*} Many general takeaways regarding MStdE are similar to
general takeaways regarding rMS(P)E: there seems to be no benefit to
using IRS, even when there is no spatial covariance; as the strength of
spatial covariance increases, the gap in MStdE between IRS-Design and
the other sampling-analysis combinations widens; and IRS-Model
outperforms IRS-Design by a large margin. This is not surprising because
all sampling-analysis combinations had nearly zero average bias, thus
rMS(P)E is driven by the variance of the estimators (design-based) or
predictors (model-based). We do note that between GRTS-Design and
GRTS-Model, GRTS-Design had lower MStdE when there was no spatial
covariance or a medium amount of spatial covariance (Fig.
\ref{fig:mse_eff}, ``Prop DE: 0'' and ``Prop DE: 0.5'' rows) and
GRTS-Model had lower MStdE when there was a high amount of spatial
covariance Fig. \ref{fig:mse_eff}, ``Prop DE: 0.9'' row). These
conclusions are similar to those observed in the grid location layout,
so we omit a grid location layout figure here. Tables for MStdE in all
36 simulation scenarios are provided in the supporting information.

\begin{figure}
  \centering
  \includegraphics[width = 1\linewidth]{figures/mse_eff.jpeg}
  \caption{Relative standard errors in the simulation study for the four sampling-analysis combinations. The rows indicate the proportion of dependent error and the columns indicate the response type.}
  \label{fig:mse_eff}
\end{figure}

Fig. \ref{fig:figconf} shows the 95\% interval coverage for each of the
four sampling-analysis combinations in the random location layout.
Within each scenario, the sampling-analysis combinations tend to have
fairly similar interval coverage. Coverage in the normal response
scenarios was usually near 95\%, while coverage in the lognormal
response scenarios varied from from 90\% to 95\% but increased with the
sample size. At a sample size of 200, all four sampling-analysis
combinations had approximately 95\% interval coverage in both response
scenarios for all dependent error proportions. These conclusions are
similar to those observed in the grid location layout, so we omit a grid
location layout figure here. Tables for interval coverage in all 36
simulation scenarios are provided in the supporting information.

\begin{figure}
  \centering
  \includegraphics[width = 1\linewidth]{figures/coverage.jpeg}
  \caption{Interval coverage in the simulation study for the four sampling-analysis combinations. The rows indicate the proportion of dependent error and the columns indicate the response type. The solid, black line represents 95\% coverage.}
  \label{fig:figconf}
\end{figure}

\hypertarget{sec:r_app}{%
\subsection{Application}\label{sec:r_app}}

Fig. \ref{fig:merc} shows a map and histogram of mercury concentration
in all 986 NLA lakes. The map shows mercury concentration exhibits some
spatial patterning, with high mercury concentrations in the northeast
and north central United States. The histogram shows that mercury
concentration is right-skewed, with most lakes having a low value of
mercury concentration but a few having a much higher concentration. Fig.
\ref{fig:merc} also shows mercury concentration's empirical
semivariogram. The empirical semivariogram can be used as a tool to
visualize spatial dependence. It quantifies the halved squared
differences (semivariance) among mercury concentration at different
distances apart. When a process has spatial covariance (exhibits spatial
dependence), the semivariance tends to be smaller at small distances and
larger at large distances. The empirical semivariogram in Fig.
\ref{fig:merc} suggests that mercury concentration is exhibits spatial
dependence. Lastly we note that the realized mean mercury concentration
in the 986 NLA lakes is 103.2 ng / g.

\begin{figure}
\centering
\begin{subfigure}{0.98\textwidth}
  \centering
  \includegraphics[width = 1\linewidth]{figures/mercury_map.jpeg}
  \caption*{}
  \label{fig:mercury_map}
\end{subfigure} \\
\begin{subfigure}{0.49\textwidth}
  \centering
  \includegraphics[width = 1\linewidth]{figures/mercury_hist.jpeg}
  \caption*{}
  \label{fig:mercury_hist}
\end{subfigure}
\begin{subfigure}{0.49\textwidth}
  \centering
  \includegraphics[width = 1\linewidth]{figures/sv_plot.jpeg}
  \caption*{}
  \label{fig:sv_plot}
\end{subfigure}
\caption{Mercury concentration visualizations for the population (Hg) for 986 lakes in the NLA data. A spatial layout is in the top row, a histogram is in the bottom row and left column, and an empirical semivariogram is in the bottom row and right column.}
\label{fig:merc}
\end{figure}

We selected a single IRS sample and a single GRTS sample and estimated
(design-based) or predicted (model-based) the mean mercury concentration
and constructed 95\% confidence (design-based) and 95\% (model-based)
prediction intervals. For the model-based analyses, the exponential
covariance was used. Table \ref{tab:appliedtab} shows the results from
these analyses. For all four sampling-analysis combinations, the true
realized mean mercury concentration is within the bounds of the 95\%
confidence (design-based) or 95\% prediction (model-based) intervals.
Though we should not generalize these results to other samples from
these data, we do note a couple of patterns. The design-based IRS
analysis shows the largest standard error: a likely reason is that this
is the only approach that does not incorporate any spatial locations.
Additionally, both analyses using GRTS sampling have lower standard
errors than both analyses using IRS sampling.

\begin{table}[ht]
\centering
\begin{tabular}{lrrrr}
  \hline
Approach & Est/Pred & SE & 95\% LB & 95\% UB \\ 
  \hline
IRS-Design & 112.7 & 8.8 & 95.4 & 129.9 \\ 
  IRS-Model & 110.5 & 7.9 & 95.0 & 125.9 \\ 
  GRTS-Design & 101.8 & 6.1 & 89.8 & 113.7 \\ 
  GRTS-Model & 102.3 & 5.9 & 90.8 & 113.9 \\ 
   \hline
\end{tabular}
\caption{\label{tab:appliedtab} For each sampling-analysis combination (Approach), estimates/predictions (Est/Pred), standard errors (SE), lower 95\% interval bounds (95\% LB), and upper 95\% interval bounds (95\% UB) for mean mercury concentration computed using a sample of 100 lakes in the NLA data. The true mean concentration of all 986 lakes in the NLA data is 103.2 ng / g.} 
\end{table}

\hypertarget{sec:discussion}{%
\section{Discussion}\label{sec:discussion}}

The design-based and model-based approaches to statistical inference are
fundamentally different paradigms. The design-based approach
incorporates randomness through sampling to estimate population
parameters. The model-based approach incorporates randomness through
distributional assumptions to predict realized values of a stochastic
process. Though these approaches have often been compared in the
literature from theoretical and analytical perspectives, our
contribution lies in studying them in a spatial context while
implementing spatially balanced sampling and the local neighborhood
variance estimator (in the design-based approach). Aside from the
theoretical differences described, a few analytical findings from the
simulation study are particularly notable. First, the sampling decision
(IRS vs GRTS) is most important when using a design-based analysis.
Though GRTS-Model still outperformed IRS-Model, the model-based analysis
mitigated most of the inefficiencies that result from the IRS samples
lacking spatial balance. Second, independent of the analysis approach,
we found no reason to prefer IRS over GRTS for sampling spatial data --
GRTS-Design and GRTS-Model generally performed at least as well as their
IRS counterparts when there was no spatial covariance and noticeably
better than their IRS counterparts when there was spatial covariance.
Third, as the strength of spatial covariance increases, the gap in
rMS(P)E between IRS-Design and the other sampling-analysis combinations
also increases. Fourth and finally, when the response was normal,
interval coverage for all sampling-analysis combinations was very close
to 95\% for all sample sizes; when the response was lognormal, interval
coverage for all sampling and analysis was between 90\% and 95\% and
closest to 95\% when \(n = 200\).

There are several benefits and drawbacks of the design-based and
model-based approaches for finite population spatial data. Some we have
discussed, but others we have not, and they are worthy of consideration
in future research. Design-based approaches are often computationally
efficient, while model-based approaches can be computationally
burdensome, especially for likelihood-based estimation methods like REML
that rely on inverting a covariance matrix. The design-based approach
also more naturally handles binary data, free from the more complicated
logistic regression framework commonly used to analyze binary data in a
model-based approach. The model-based approach, however, can more
naturally quantify the relationship between covariates (predictor
variables) and response variable. The model-based approach also yields
estimated spatial covariance parameters, which help better understand
the dependence structure in the stochastic process of study. Model
selection is also possible using model-based approaches and criteria
such as cross validation, likelihood ratio tests, or AIC (Akaike, 1974).
Model-based approaches are capable of more efficient small-area
estimation than design-based approaches by leveraging distributional
assumptions in areas with few observed sites. Model-based approaches can
also compute site-by-site predictions at unobserved locations and use
them to construct informative visualizations like smoothed maps. In
short, when deciding whether the design-based or model-based approach is
more appropriate to implement, the benefits and drawbacks of each
approach should be considered alongside the particular goals of the
study.

\hypertarget{acknowledgments}{%
\section*{Acknowledgments}\label{acknowledgments}}
\addcontentsline{toc}{section}{Acknowledgments}

The views expressed in this manuscript are those of the authors and do
not necessarily represent the views or policies of the U.S.
Environmental Protection Agency or the National Oceanic and Atmospheric
Administration. Any mention of trade names, products, or services does
not imply an endorsement by the U.S. government, the U.S. Environmental
Protection Agency, or the National Oceanic and Atmospheric
Administration. The U.S. Environmental Protection Agency and National
Oceanic and Atmospheric Administration do not endorse any commercial
products, services, or enterprises.

\hypertarget{conflict-of-interest-statement}{%
\section*{Conflict of Interest
Statement}\label{conflict-of-interest-statement}}
\addcontentsline{toc}{section}{Conflict of Interest Statement}

There are no conflicts of interest for any of the authors.

\hypertarget{author-contribution-statement}{%
\section*{Author Contribution
Statement}\label{author-contribution-statement}}
\addcontentsline{toc}{section}{Author Contribution Statement}

All authors conceived the ideas; All authors designed methodology; MD
and MH performed the simulations and analyzed the data; MD and MH led
the writing of the manuscript; All authors contributed critically to the
drafts and gave final approval for publication.

\hypertarget{data-and-code-availability}{%
\section*{Data and Code Availability}\label{data-and-code-availability}}
\addcontentsline{toc}{section}{Data and Code Availability}

This manuscript has a supplementary R package that contains all of the
data and code used in its creation. The supplementary R package is
hosted on GitHub. Instructions for download at available at

\url{https://github.com/michaeldumelle/DvMsp}.

If the manuscript is accepted, this repository will be archived in
Zenodo.

\hypertarget{supporting-information}{%
\section*{Supporting Information}\label{supporting-information}}
\addcontentsline{toc}{section}{Supporting Information}

In the supporting information, we provide tables of summary statistics
for all 36 simulation scenarios.

\hypertarget{references}{%
\section*{References}\label{references}}
\addcontentsline{toc}{section}{References}

\hypertarget{refs}{}
\leavevmode\hypertarget{ref-akaike1974new}{}%
Akaike, H., 1974. A new look at the statistical model identification.
IEEE Transactions on Automatic Control 19, 716--723.

\leavevmode\hypertarget{ref-barabesi2011sampling}{}%
Barabesi, L., Franceschi, S., 2011. Sampling properties of spatial total
estimators under tessellation stratified designs. Environmetrics 22,
271--278.

\leavevmode\hypertarget{ref-benedetti2017spatially}{}%
Benedetti, R., Piersimoni, F., 2017. A spatially balanced design with
probability function proportional to the within sample distance.
Biometrical Journal 59, 1067--1084.

\leavevmode\hypertarget{ref-benedetti2017spatiallyreview}{}%
Benedetti, R., Piersimoni, F., Postiglione, P., 2017. Spatially balanced
sampling: A review and a reappraisal. International Statistical Review
85, 439--454.

\leavevmode\hypertarget{ref-breiman2001random}{}%
Breiman, L., 2001. Random forests. Machine Learning 45, 5--32.

\leavevmode\hypertarget{ref-brus1997random}{}%
Brus, D., De Gruijter, J., 1997. Random sampling or geostatistical
modelling? Choosing between design-based and model-dased sampling
strategies for soil (with discussion). Geoderma 80, 1--44.

\leavevmode\hypertarget{ref-brus2021statistical}{}%
Brus, D.J., 2021. Statistical approaches for spatial sample survey:
Persistent misconceptions and new developments. European Journal of Soil
Science 72, 686--703.

\leavevmode\hypertarget{ref-chan2020bayesian}{}%
Chan-Golston, A.M., Banerjee, S., Handcock, M.S., 2020. Bayesian
inference for finite populations under spatial process settings.
Environmetrics 31, e2606.

\leavevmode\hypertarget{ref-chiles1999geostatistics}{}%
Chiles, J.-P., Delfiner, P., 1999. Geostatistics: Modeling Spatial
Uncertainty. John Wiley \& Sons, New York.

\leavevmode\hypertarget{ref-cicchitelli2012model}{}%
Cicchitelli, G., Montanari, G.E., 2012. Model-assisted estimation of a
spatial population mean. International Statistical Review 80, 111--126.

\leavevmode\hypertarget{ref-cooper2006sampling}{}%
Cooper, C., 2006. Sampling and variance estimation on continuous
domains. Environmetrics 17, 539--553.

\leavevmode\hypertarget{ref-cressie1993statistics}{}%
Cressie, N., 1993. Statistics for spatial data. John Wiley \& Sons.

\leavevmode\hypertarget{ref-de1990model}{}%
De Gruijter, J., Ter Braak, C., 1990. Model-free estimation from spatial
samples: A reappraisal of classical sampling theory. Mathematical
Geology 22, 407--415.

\leavevmode\hypertarget{ref-diggle2010geostatistical}{}%
Diggle, P.J., Menezes, R., Su, T.-l., 2010. Geostatistical inference
under preferential sampling. Journal of the Royal Statistical Society:
Series C (Applied Statistics) 59, 191--232.

\leavevmode\hypertarget{ref-dumelle2021spsurvey}{}%
Dumelle, M., Kincaid, T.M., Olsen, A.R., Weber, M.H., 2021. Spsurvey:
Spatial sampling design and analysis.

\leavevmode\hypertarget{ref-fix1989discriminatory}{}%
Fix, E., Hodges, J.L., 1989. Discriminatory analysis. Nonparametric
discrimination: Consistency properties. International Statistical
Review/Revue Internationale de Statistique 57, 238--247.

\leavevmode\hypertarget{ref-foster2017spatially}{}%
Foster, S.D., Hosack, G.R., Lawrence, E., Przeslawski, R., Hedge, P.,
Caley, M.J., Barrett, N.S., Williams, A., Li, J., Lynch, T., others,
2017. Spatially balanced designs that incorporate legacy sites. Methods
in Ecology and Evolution 8, 1433--1442.

\leavevmode\hypertarget{ref-grafstrom2012spatiallypoisson}{}%
Grafström, A., 2012. Spatially correlated poisson sampling. Journal of
Statistical Planning and Inference 142, 139--147.

\leavevmode\hypertarget{ref-grafstrom2013well}{}%
Grafström, A., Lundström, N.L., 2013. Why well spread probability
samples are balanced. Open Journal of Statistics 3, 36--41.

\leavevmode\hypertarget{ref-grafstrom2012spatially}{}%
Grafström, A., Lundström, N.L., Schelin, L., 2012. Spatially balanced
sampling through the pivotal method. Biometrics 68, 514--520.

\leavevmode\hypertarget{ref-grafstrom2018spatially}{}%
Grafström, A., Matei, A., 2018. Spatially balanced sampling of
continuous populations. Scandinavian Journal of Statistics 45, 792--805.

\leavevmode\hypertarget{ref-hansen1983evaluation}{}%
Hansen, M.H., Madow, W.G., Tepping, B.J., 1983. An evaluation of
model-dependent and probability-sampling inferences in sample surveys.
Journal of the American Statistical Association 78, 776--793.

\leavevmode\hypertarget{ref-harville1977maximum}{}%
Harville, D.A., 1977. Maximum likelihood approaches to variance
component estimation and to related problems. Journal of the American
Statistical Association 72, 320--338.

\leavevmode\hypertarget{ref-higham2021sptotal}{}%
Higham, M., Ver Hoef, J., Frank, B., Dumelle, M., 2021. Sptotal:
Predicting totals and weighted sums from spatial data.

\leavevmode\hypertarget{ref-horvitz1952generalization}{}%
Horvitz, D.G., Thompson, D.J., 1952. A generalization of sampling
without replacement from a finite universe. Journal of the American
Statistical Association 47, 663--685.

\leavevmode\hypertarget{ref-lohr2009sampling}{}%
Lohr, S.L., 2009. Sampling: Design and analysis. Nelson Education.

\leavevmode\hypertarget{ref-patterson1971recovery}{}%
Patterson, H.D., Thompson, R., 1971. Recovery of inter-block information
when block sizes are unequal. Biometrika 58, 545--554.

\leavevmode\hypertarget{ref-robertson2013bas}{}%
Robertson, B., Brown, J., McDonald, T., Jaksons, P., 2013. BAS: Balanced
acceptance sampling of natural resources. Biometrics 69, 776--784.

\leavevmode\hypertarget{ref-robertson2018halton}{}%
Robertson, B., McDonald, T., Price, C., Brown, J., 2018. Halton
iterative partitioning: Spatially balanced sampling via partitioning.
Environmental and Ecological Statistics 25, 305--323.

\leavevmode\hypertarget{ref-sarndal2003model}{}%
Särndal, C.-E., Swensson, B., Wretman, J., 2003. Model assisted survey
sampling. Springer Science \& Business Media.

\leavevmode\hypertarget{ref-schabenberger2017statistical}{}%
Schabenberger, O., Gotway, C.A., 2017. Statistical methods for spatial
data analysis. CRC press.

\leavevmode\hypertarget{ref-sen1953estimate}{}%
Sen, A.R., 1953. On the estimate of the variance in sampling with
varying probabilities. Journal of the Indian Society of Agricultural
Statistics 5, 127.

\leavevmode\hypertarget{ref-sterba2009alternative}{}%
Sterba, S.K., 2009. Alternative model-based and design-based frameworks
for inference from samples to populations: From polarization to
integration. Multivariate Behavioral Research 44, 711--740.

\leavevmode\hypertarget{ref-stevens2003variance}{}%
Stevens, D.L., Olsen, A.R., 2003. Variance estimation for spatially
balanced samples of environmental resources. Environmetrics 14,
593--610.

\leavevmode\hypertarget{ref-stevens2004spatially}{}%
Stevens, D.L., Olsen, A.R., 2004. Spatially balanced sampling of natural
resources. Journal of the American Statistical Association 99, 262--278.

\leavevmode\hypertarget{ref-USEPA2012NLA}{}%
USEPA, 2012. National lakes assessment 2012.
https://www.epa.gov/national-aquatic-resource-surveys/national-results-and-regional-highlights-national-lakes-assessment.

\leavevmode\hypertarget{ref-verhoef2002sampling}{}%
Ver Hoef, J., 2002. Sampling and geostatistics for spatial data.
Ecoscience 9, 152--161.

\leavevmode\hypertarget{ref-verhoef2008spatial}{}%
Ver Hoef, J.M., 2008. Spatial methods for plot-based sampling of
wildlife populations. Environmental and Ecological Statistics 15, 3--13.

\leavevmode\hypertarget{ref-ver2013comparison}{}%
Ver Hoef, J.M., Temesgen, H., 2013. A comparison of the spatial linear
model to nearest neighbor (k-nn) methods for forestry applications. PlOS
ONE 8, e59129.

\leavevmode\hypertarget{ref-wang2013design}{}%
Wang, J.-F., Jiang, C.-S., Hu, M.-G., Cao, Z.-D., Guo, Y.-S., Li, L.-F.,
Liu, T.-J., Meng, B., 2013. Design-based spatial sampling: Theory and
implementation. Environmental Modelling \& Software 40, 280--288.

\leavevmode\hypertarget{ref-wang2012review}{}%
Wang, J.-F., Stein, A., Gao, B.-B., Ge, Y., 2012. A review of spatial
sampling. Spatial Statistics 2, 1--14.

\leavevmode\hypertarget{ref-wolfinger1994computing}{}%
Wolfinger, R., Tobias, R., Sall, J., 1994. Computing gaussian
likelihoods and their derivatives for general linear mixed models. SIAM
Journal on Scientific Computing 15, 1294--1310.


\end{document}


