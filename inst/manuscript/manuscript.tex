\documentclass[]{elsarticle} %review=doublespace preprint=single 5p=2 column
%%% Begin My package additions %%%%%%%%%%%%%%%%%%%
\usepackage[hyphens]{url}

  \journal{Methods in Ecology and Evolution} % Sets Journal name


\usepackage{lineno} % add
  \linenumbers % turns line numbering on
\providecommand{\tightlist}{%
  \setlength{\itemsep}{0pt}\setlength{\parskip}{0pt}}

\usepackage{graphicx}
\usepackage{booktabs} % book-quality tables
%%%%%%%%%%%%%%%% end my additions to header

\usepackage[T1]{fontenc}
\usepackage{lmodern}
\usepackage{amssymb,amsmath}
\usepackage{ifxetex,ifluatex}
\usepackage{fixltx2e} % provides \textsubscript
% use upquote if available, for straight quotes in verbatim environments
\IfFileExists{upquote.sty}{\usepackage{upquote}}{}
\ifnum 0\ifxetex 1\fi\ifluatex 1\fi=0 % if pdftex
  \usepackage[utf8]{inputenc}
\else % if luatex or xelatex
  \usepackage{fontspec}
  \ifxetex
    \usepackage{xltxtra,xunicode}
  \fi
  \defaultfontfeatures{Mapping=tex-text,Scale=MatchLowercase}
  \newcommand{\euro}{€}
\fi
% use microtype if available
\IfFileExists{microtype.sty}{\usepackage{microtype}}{}
\bibliographystyle{elsarticle-harv}
\ifxetex
  \usepackage[setpagesize=false, % page size defined by xetex
              unicode=false, % unicode breaks when used with xetex
              xetex]{hyperref}
\else
  \usepackage[unicode=true]{hyperref}
\fi
\hypersetup{breaklinks=true,
            bookmarks=true,
            pdfauthor={},
            pdftitle={A comparison of design-based and model-based approaches for finite population spatial data.},
            colorlinks=false,
            urlcolor=blue,
            linkcolor=magenta,
            pdfborder={0 0 0}}
\urlstyle{same}  % don't use monospace font for urls

\setcounter{secnumdepth}{5}
% Pandoc toggle for numbering sections (defaults to be off)

% Pandoc citation processing

% Pandoc header

\usepackage{bm} \usepackage{bbm} \usepackage{color} \DeclareMathOperator{\var}{{var}} \DeclareMathOperator{\cov}{{cov}}

\begin{document}
\begin{frontmatter}

  \title{A comparison of design-based and model-based approaches for finite
population spatial data.}
    \author[USEPA]{Michael Dumelle\corref{1}}
  
    \author[STLAW]{Matt Higham}
  
    \author[NOAA]{Jay M. Ver Hoef}
  
    \author[USEPA]{Anthony R. Olsen}
  
    \author[OSU]{Lisa Madsen}
  
      \address[USEPA]{United States Environmental Protection Agency, 200 SW 35th St,
Corvallis, Oregon, 97333}
    \address[STLAW]{Saint Lawrence University Department of Mathematics, Computer Science,
and Statistics, 23 Romoda Drive, Canton, New York, 13617}
    \address[NOAA]{Marine Mammal Laboratory, Alaska Fisheries Science Center, National
Oceanic and Atmospheric Administration, Seattle, Washington, 98115}
    \address[OSU]{Oregon State University Department of Statistics, 239 Weniger Hall,
Corvallis, Oregon, 97331}
      \cortext[1]{Corresponding Author: Michael Dumelle (Dumelle.Michael@epa.gov)}
  
  \begin{abstract}
  The design-based and model-based approaches to frequentist statistical
  inference lie on fundamentally different foundations. In the
  design-based approach, inference depends on random sampling. In the
  model-based approach, inference depends on distributional assumptions.
  In this manuscript, we compare the approaches for finite population
  spatial data. We first provide relevant background for the approaches
  and then use a simulation study and an analysis of real mercury
  concentration data to numerically compare them. We find that sampling
  plans that incorporate spatial locations (spatially balanced samples)
  perform better than sampling plans ignoring spatial locations
  (non-spatially balanced samples), regardless of whether design-based or
  model-based approaches were used to analyze the data. We also find that
  within sampling plans, the model-based approaches often outperform
  design-based approaches, even for skewed data. This gap in performance
  is small when spatially balanced samples are used but large when
  non-spatially balanced samples are used.
  \end{abstract}
  
 \end{frontmatter}

\hypertarget{sec:introduction}{%
\section{Introduction}\label{sec:introduction}}

There are two general approaches for using data to make frequentist
statistical inferences about a population: design-based and model-based.
When data cannot be collected for all units in a population (i.e.,
population units), data are collected on a subset of the population
units. This subset is called a sample. In the design-based approach,
inferences about the underlying population are informed via a
probabilistic process assigning some population units to be part of the
sample. Alternatively, in the model-based approach, inferences are made
from specific assumptions about the underlying process generating the
data. Each paradigm has a deep historical context (Sterba, 2009) and its
own set of benefits and drawbacks (Hansen et al., 1983).

Though the design-based and model-based approaches apply to statistical
inference in a broad sense, we focus on comparing these approaches for
spatial data. We define spatial data as data that incorporates the
specific locations of the population units into either the design or
estimation process. De Gruijter and Ter Braak (1990) give an early
comparison of design-based and model-based approaches for spatial data,
quashing the belief that design-based approaches could not be used for
spatially correlated data. Since then, there have been several general
comparisons between design-based and model-based approaches for spatial
data (Brus and De Gruijter, 1997; Brus, 2021; Ver Hoef, 2002, 2008; Wang
et al., 2012). Cooper (2006) reviews the two approaches in an ecological
context before introducing a ``model-assisted'' variance estimator that
combines aspects from each approach. In addition to Cooper (2006), there
has been substantial research and development into estimators that use
both design and model-based principles (see e.g., Sterba (2009),
Cicchitelli and Montanari (2012), Chan-Golston et al. (2020) for a
Bayesian approach).

Certainly comparisons between design-based and model-based approaches to
spatial data have been studied. But no numerical comparison has been
made between design-based approaches incorporating spatial information
and design-based approaches. In this manuscript, we compare design-based
approaches incorporating spatial information to model-based approaches
for spatial data. We focus on finite populations, but these comparisons
generalize to infinite populations as well. A finite population contains
a finite number of population units; an example is lakes (treated as a
whole with the lake centroid representing location) in the contiguous
United States. An infinite population contains an infinite number of
population units; an example is locations within a single lake.

The rest of the manuscript is organized as follows. In Section
\ref{sec:background}, we introduce and provide relevant background for
the design-based and model-based approaches to finite population spatial
data. In Section \ref{sec:numstudy}, we use a simulation study to
compare the performance of the approaches in a variety of scenarios. In
Section \ref{application}, we compare the performance of the approaches
on real data that contains mercury concentration in lakes from the
contiguous United States. And in Section \ref{sec:discussion}, we end
with a discussion and provide directions for future research.

\hypertarget{sec:background}{%
\section{Background}\label{sec:background}}

The design-based and model-based approaches incorporate randomness in
fundamentally different ways. In this section, we describe the role of
randomness for each approach and the subsequent effects on statistical
inferences for spatial data.

\hypertarget{subsec:dvm_compare}{%
\subsection{Comparing Design-Based and Model-Based
Approaches}\label{subsec:dvm_compare}}

The design-based approach assumes the population is fixed. Randomness is
incorporated via the selection of units in a sampling frame. A sampling
frame is the set of all units available to be sampled. Units from the
sampling frame are selected as part of the sample according to a
sampling design, which assigns a positive probability of inclusion
(inclusion probability) to each unit from the sampling frame. Some
examples of commonly used sampling designs include simple random
sampling, stratified random sampling, and cluster sampling. When
sampling designs incorporate spatial locations into sampling, we call
the resulting samples ``spatially balanced.'' One approach to selecting
spatially balanced samples is the Generalized Random Tessellation
Stratified (GRTS) algorithm (Stevens and Olsen, 2004), which we discuss
in more detail in Section \ref{subsec:spb_design}. When sampling designs
do not incorporate spatial locations into sampling, we call the
resulting samples ``non-spatially balanced.''

Fundamentally, the design-based approach combines the randomness of the
sampling design with the data collected via the sample to justify the
estimation and uncertainty quantification of fixed, unknown parameters
of a population (e.g., a population mean). Treating the data as fixed
and incorporating randomness through the sampling design yields
estimators having very few other assumptions. Confidence intervals for
these types of estimators are typically derived using limiting arguments
that incorporate all possible samples. Sample means, for example, are
asymptotically normal (Gaussian) by the Central Limit Theorem (under
some assumptions). If we repeatedly select samples from the population,
then 95\% of all 95\% confidence intervals constructed from a procedure
with appropriate coverage will contain the true, fixed mean. Särndal et
al. (2003) and Lohr (2009) provide thorough reviews of the design-based
approach.

The model-based approach assumes the data are a random realization of a
data-generating stochastic process. Randomness is incorporated through
distributional assumptions on this process. Strictly speaking,
randomness need not be incorporated through random sampling, though
Diggle et al. (2010) warn against preferential sampling. Preferential
sampling occurs when the process generating the data locations and the
process being modeled are not independent of one another. To guard
against preferential sampling, model-based approaches often still
implement some form of random sampling.

Instead of estimating fixed, unknown population parameters, as in the
design-based approach, often the goal of model-based inference is to
predict a realized variable, or value. For example, suppose the realized
mean of all population units is the value of interest. Instead of
\emph{estimating} a fixed, unknown mean, we are \emph{predicting} the
value of the mean, a random variable. Prediction intervals are then
derived using assumptions of the data-generating stochastic process. If
we repeatedly generate response values from the same data-generating
stochastic process and select samples, then 95\% of all 95\% prediction
intervals constructed from a procedure with appropriate coverage will
contain their respective realized means. Cressie (1993) and
Schabenberger and Gotway (2017) provide thorough reviews of model-based
approaches for spatial data. In Figure \ref{fig:fig1}, we provide a
visual comparison of the design-based and model-based approaches (Ver
Hoef (2002) and Brus (2021) provide similar figures).

\begin{figure}
\includegraphics[width=1\linewidth]{manuscript_files/figure-latex/fig1-1} \caption{A visual comparison of the design-based and model-based approaches. In the top row, there is one fixed population with nine population units and three random samples of size four (points circled are those sampled). The response values at each site are fixed, but we obtain different estimates for the mean response in each random sample. In the bottom row, there are three realizations of the same data-generating stochastic process that are all sampled at the same four locations. The ata-generating stochastic process has a single mean, but the mean of the nine population units is different in each of the three realizations.}\label{fig:fig1}
\end{figure}

\hypertarget{subsec:spb_design}{%
\subsection{Spatially Balanced Design and
Analysis}\label{subsec:spb_design}}

We previously mentioned that the design-based approach can be used to
select spatially balanced samples (samples that incorporate spatial
locations of the population units and are ``well-spread'' is space).
Spatially balanced samples are useful because parameter estimates from
these samples tend to vary less than parameter estimates from samples
that are not spatially balanced (Barabesi and Franceschi, 2011;
Benedetti et al., 2017; Grafström and Lundström, 2013; Robertson et al.,
2013; Stevens and Olsen, 2004; Wang et al., 2013). The first spatially
balanced sampling algorithm seeing widespread use is the Generalized
Random Tessellation Stratified (GRTS) algorithm (Stevens and Olsen,
2004). To quantify the spatial balance of a sample, Stevens and Olsen
(2004) proposed loss metrics based on Voronoi polygons (Direchlet
Tessellations). After the GRTS algorithm was developed, several other
spatially balanced sampling algorithms emerged, such as the Local
Pivotal Method (Grafström et al., 2012; Grafström and Matei, 2018),
Spatially Correlated Poisson Sampling (Grafström, 2012), Balanced
Acceptance Sampling (Robertson et al., 2013), Within-Sample-Distance
Sampling (Benedetti and Piersimoni, 2017), and Halton Iterative
Partitioning Sampling (Robertson et al., 2018). In this manuscript, we
select spatially balanced samples using the Generalized Random
Tessellation Stratified (GRTS) algorithm because it has several
attractive properties. More specifically, the GRTS algorithm
accommodates finite and infinite sampling frames, equal, unequal, and
proportional (to size) inclusion probabilities, legacy (historical)
sampling (Foster et al., 2017), a minimum distance between units in a
sample, and replacement units (replacement units are population units
that can be sampled when a population unit originally selected can no
longer be sampled). The GRTS algorithm selects samples by utilizing a
particular mapping between two-dimensional and one-dimensional space
that preserves proximity relationships. Via this mapping, units in
two-dimensional space are partitioned using a hierarchical address. This
hierarchical address is used to map population units to a
one-dimensional line. On the one dimensional line, each population
unit's line length equals its inclusion probability. Then, a systematic
sample of population units is selected on the line, yielding desired
sample. Stevens and Olsen (2004) provides more technical details.

After selecting a sample and collecting data, unbiased estimates of
population means and totals can be obtained using the Horvitz-Thompson
estimator (Horvitz and Thompson, 1952). If \(\tau\) is a population
total, the Horvitz-Thompson estimate of \(\tau\), denoted by
\(\hat{\tau}_{ht}\), is is given by \begin{align}\label{eq:ht}
  \hat{\tau}_{ht} = \sum_{i = 1}^n Z_i \pi_i^{-1},
\end{align} where \(Z_i\) is the value of the \(i\)th population unit in
the sample and \(\pi_i\) is the inclusion probability of the \(i\)th
population unit in the sample. An estimate of the population mean is
obtained by dividing \(\hat{\tau}_{ht}\) by \(N\), the number of
population units.

It is also important to quantify uncertainty \(\hat{\tau}_{ht}\).
Horvitz and Thompson (1952) and Sen (1953) provide variance estimators
for \(\hat{\tau}_{ht}\), but these estimators have two drawbacks. First,
they rely on calculating \(\pi_{ij}\), the probability that population
unit \(i\) and populatoin unit \(j\) are both in the sample -- this
quantity can be challenging if not impossible to calculate analytically.
Second, these estimators ignore the spatial locations of the population
units. To address these two drawbacks simultaneously, Stevens and Olsen
(2003) proposed the local neighborhood variance estimator. The local
neighborhood variance estimator does not rely on \(\pi_{ij}\) and
incorporates spatial locations -- for technical details see Stevens and
Olsen (2003). Stevens and Olsen (2003) show the local neighborhood
variance estimator tends to reduce the estimated variance of
\(\hat{\tau}\) and yield narrower confidence intervals compared to
variance estimators that ignore spatial locations.

\hypertarget{finite-population-block-kriging}{%
\subsection{Finite Population Block
Kriging}\label{finite-population-block-kriging}}

Finite Population Block Kriging (FPBK) is a model-based approach that
expands the geostatistical Kriging framework to the finite population
setting (Ver Hoef, 2008). Instead of developing inference based on a
specific sampling design, we assume the data are generated by a spatial
stochastic process. We summarize some of the basic principles of FBPK
next (for more technical details, see Ver Hoef (2008)) Let
\({\mathbf{z} \equiv \{\text{z}(s_1), \text{z}(s_2), . . . , \text{z}(s_N) \}}\)
be an \(N \times 1\) response vector at locations \(s_1\), \(s_2\), . .
. , \(s_N\) that can be measured at the \(N\) population units. Suppose
we want to use a sample to predict some linear function of the response
variable, \(f(\mathbf{z}) = \mathbf{b}^\prime \mathbf{z}\), where
\(\mathbf{b}^\prime\) is a \(1 \times N\) vector of weights (e.g, the
population mean is represented by a weights vector whose elements all
equal one). Denoting quantities that are part of the sampled population
units with a subscript \emph{s} and quantities that are part of the
unsampled population units with subscript \emph{u}, let

\begin{equation}
\begin{pmatrix} \label{equation:Zmarginal}
    \mathbf{z}_s      \\
    \mathbf{z}_u
\end{pmatrix}
=
\begin{pmatrix}
  \mathbf{X}_s    \\
  \mathbf{X}_u
\end{pmatrix}
\bm{\beta} +
\begin{pmatrix}
\bm{\delta}_s    \\
\bm{\delta}_u
\end{pmatrix},
\end{equation} where \(\mathbf{X}_s\) and \(\mathbf{X}_u\) are the
design matrices for the sampled and unsampled population units,
respectively, \(\bm{\beta}\) is the parameter vector of fixed effects,
and \(\bm{\delta} \equiv [\bm{\delta}_s \,\, \bm{\delta}_u]'\), where
\(\bm{\delta}_s\) and \(\bm{\delta}_u\) are random errors for the
sampled and unsampled population units, respectively.

FBPK assumes \(\bm{\delta}\) in Equation\(~\)\ref{equation:Zmarginal}
has mean zero and a spatial correlation structure that can be modeled
using a covariance function. This covariance function is commonly
assumed to be non-negative (between zero and one), second-order
stationary (depending only on the distance between population units),
isotropic (independent of direction), and decay with distance between
population units (Cressie, 1993). Henceforth, it is implied that we have
made these same assumptions regarding \(\bm{\delta}\), though Chiles and
Delfiner (1999), pp.~80-93 discuss covariance functions that are not
second-order stationary, not isotropic, or both. A variety of flexible
covariance functions can be used to model \(\bm{\delta}\) (Cressie,
1993); one example is the exponential covariance function (for a
thorough list of spatial covariance functions, see Cressie (1993). The
\(i,j\)th element of the exponential covariance matrix,
\(\mathop{\mathrm{{cov}}}(\bm{\delta})\), is \mbox{}
\begin{align}\label{equation:expcov}
\mathop{\mathrm{{cov}}}(\delta_i, \delta_j) = 
\begin{cases} 
\sigma^2_{1}\exp(-h_{i,j}/\phi) & h_{i,j} > 0 \\
\sigma^2_{1} + \sigma^2_2 & h_{i,j} = 0
\end{cases}
,
\end{align} where \(\sigma^2_{1}\) is the variance parameter quantifying
the variability that is dependent (coarse-scale), \(\sigma^2_{2}\) is
the variance parameter quantifying the variability that is independent
(fine-scale), \(\phi\) is the range parameter measuring the
distance-decay rate of the covariance, and \(h_{i,j}\) is the Euclidean
distance between population units \(i\) and \(j\). The proportion of
variability attributable to dependent random error is
\(\sigma^2_{1} / (\sigma^2_{1} + \sigma^2_{2})\). Similarly, the
proportion of variability attributable to independent random error is
\(\sigma^2_{2} / (\sigma^2_{1} + \sigma^2_{2})\). Finally we note that
\(\sigma^2_{1}\) and \(\sigma^2_{2}\) are often called the partial sill
and nugget, respectively.

With the above model formulation, the Best Linear Unbiased Predictor
(BLUP) for \(f(\mathbf{b}'\mathbf{z})\) and its prediction variance can
be computed. While details of the derivation are in Ver Hoef (2008), we
note here that the predictor and its variance are both moment-based,
meaning that they do not rely on any distributional assumptions.

Other approaches, such as k-nearest-neighbors (Fix and Hodges, 1989; Ver
Hoef and Temesgen, 2013), random forests (Breiman, 2001), Bayesian
models (Chan-Golston et al., 2020), among others, could also be used to
obtain predictions for a mean or total from spatially correlated
responses of a finite population. Compared to the k-nearest-neighbors
and random forest approach, we prefer FBPK because it is model-based and
relies on theoretically-based variance estimators leveraging the model's
spatial covariance structure, whereas k-nearest-neighbors and random
forests use ad-hoc variance estimators (Ver Hoef and Temesgen, 2013).
Additionally, Ver Hoef and Temesgen (2013) studied compared FBPK,
k-nearest-neighbors, and random forests in a variety of spatial data
contexts, and FBPK tended to perform best. Compared to the Bayesian
approach, we prefer FPBK mostly because it is much more computationally
efficient.

\hypertarget{sec:numstudy}{%
\section{Numerical Study}\label{sec:numstudy}}

We used a simulation study to investigate performance of four
sampling-analysis combinations: IRS-Design, IRS with a design-based
analysis; IRS-Model, IRS with a model-based analysis; GRTS-Design, GRTS
sampling with a design-based analysis; and GRTS-Model, GRTS sampling
with a model-based analysis. These combinations are also provided in
Table \ref{tab:designanalysis}.

\begin{table}[ht]
\centering
\begin{tabular}{r|ll}
  \hline
 & Design & Model \\ 
  \hline
IRS & IRS-Design & IRS-Model \\ 
  GRTS & GRTS-Design & GRTS-Model \\ 
   \hline
\end{tabular}
\caption{\label{tab:designanalysis} Sampling-analysis combinations in the simulation study. The rows give the two types of sampling designs and the columns give the two types of analyses.} 
\end{table}

Performance of the four sampling-analysis combinations was evaluated in
36 different simulation scenarios. The 36 scenarios resulted from the
crossing of three sample sizes, two location layouts, two response
types, and three proportions of dependent random error. The three sample
sizes (\(n\)) were \(n = 50, n = 100,\) and \(n = 200\). Samples were
always selected from a population size (\(N\)) of \(N = 900\). The two
location layouts were random and gridded. Locations in the random layout
were selected randomly from the unit square (\([0, 1] \times [0, 1]\)).
Locations in the gridded layout were selected randomly on a fixed grid
from the unit square. The two response types were normal and lognormal.
For the normal response type, the response was simulated using mean-zero
random errors with the exponential covariance
(Equation\(~\)\ref{equation:expcov}) for varying proportions of
dependent random error. The proportion of dependent random error is
represented by \(\sigma^2_1 / (\sigma^2_1 + \sigma^2_2)\), where
\(\sigma^2_1\) and \(\sigma^2_2\) are from
Equation\(~\)\ref{equation:expcov}. The total variance,
\(\sigma^2_1 + \sigma^2_2\), was always 2. The range was always
\(\sqrt{2} / 3\), which means that the correlation in the dependent
random error decayed to nearly zero at the largest possible distance
between two units in the domain. For the lognormal response type, the
response was first simulated using the same approach as for the normal
response type, except that the total variance was 0.6931 instead of 2.
The response was then exponentiated, yielding a random variable whose
total variance is 2. The lognormal responses were used to evaluate
performance of the sampling-analysis approaches for data that were
skewed.

\begin{table}[ht]
\centering
\begin{tabular}{r|lll}
   \hline
Sample Size (n) & 50 & 100 & 200 \\ 
  Location Layout & Random & Gridded & - \\ 
  Proportion of Dependent Error & 0 & 0.5 & 0.9 \\ 
  Response Type & Normal & Lognormal & - \\ 
   \hline
\end{tabular}
\caption{\label{tab:parmtab} Simulation scenario options. All combinations of sample size, location layout, response type, and proportion of dependent random error composed the 36 simulation scenarios. In each simualtion scenario, the total variance was two.} 
\end{table}

In each of the 36 simulation scenarios, there were 2000 independent
simulation trials. In each trial, IRS and GRTS samples were selected and
then design-based and model-based analyses were used to estimate the
mean and construct confidence (design-based) or prediction (model-based)
intervals. We recorded the bias, squared error, and interval coverage
for all sampling-analysis combinations in each trial. Then we summarized
the performance of the combinations across trials by calculating average
bias, rMS(P)E (root-mean-squared error for the design-based approaches
and root-mean-squared-prediction error for the model-based approaches),
and the rate at which the true mean is contained in its 95\% interval.
The GRTS algorithm and the local neighborhood variance estimator are
available in the \textbf{\textsf{R}} package \texttt{spsurvey} (Dumelle
et al., 2021). FPBK is available in the \texttt{sptotal}
\textbf{\textsf{R}} package (Higham et al., 2021) and covariance
parameters were estimated using Restricted Maximum Likelihood (Harville,
1977; Patterson and Thompson, 1971; Wolfinger et al., 1994).

The average bias was nearly zero for all four combinations in all 36
scenarios, so we omit a more detailed summary of those results here.
Tables for average bias in all 36 simulation scenarios are provided in
the supplementary material.

Figure \ref{fig:figeff} shows the relative rMS(P)E of the four
approaches from Table \ref{tab:designanalysis} using the random location
layout with ``IRS-Design'' as the baseline. More formally, the relative
rMS(P)E is defined as \begin{equation*}
\frac{\text{rMS(P)E of sampling-analysis combination}}{\text{rMS(P)E of IRS-Design}},
\end{equation*} When there is no spatial correlation (Figure
\ref{fig:figeff}, top row), the four sampling-analysis combinations have
approximately equal rMS(P)E. So, using GRTS or using a spatial model
does not result in much, if any, loss in efficiency even when the
response variable is not spatially correlated. When there is spatial
correlation (Figure \ref{fig:figeff}, middle and bottom row), the
GRTS-Model combination tends to perform best, followed by GRTS-Design,
IRS-Model, and finally IRS-Design, though the difference in relative
rMS(P)E among IRS-Model, GRTS-Design, and GRTS-Model is relatively
small. As the strength of spatial correlation increases, the gap in
rMS(P)E between IRS-Design and the other combinations widens. Finally we
note that when there is spatial correlation, IRS-Model outperforms
IRS-Design by a large margin, suggesting that the poor design properties
of IRS are largely mitigated by the model-based analysis. These
conclusions are similar to those observed in the grid location layout.
Tables for rMS(P)E in all 36 simulation scenarios are provided in the
supplementary material.

\begin{figure}
\includegraphics[width=1\linewidth]{manuscript_files/figure-latex/figeff-1} \caption{Relative rMS(P)E for the four sampling-analysis combinations. The rows indicate the proportion of dependent error and the columns indicate the response type.}\label{fig:figeff}
\end{figure}

We also studied 95\% interval coverage among the combinations. The
design-based 95\% confidence intervals and model-based 95\% prediction
intervals were constructed using the normal distribution. Justification
for the design-based and model-based intervals comes from the asymptotic
normality of totals via the Central Limit Theorem.

Figure \ref{fig:figconf} shows the 95\% interval coverage for each of
the four combinations in the random location layout. All four
combinations have fairly similar interval coverage within each scenario.
Coverage in the normal response scenarios tended to be near 95\% and
slightly higher than coverage in the lognormal scenarios. Coverage in
the lognormal scenarios still generally exceeded 90\%. Coverage tended
to always increase with the sample size. At a sample size of 200, all
four combinations had approximately 95\% interval coverage in both
response scenarios and all dependent error proportions. These
conclusions were similar to those found in the grid location layout.
Tables for interval coverage in all 36 simulation scenarios are provided
in the supplementary material.

\begin{figure}
\includegraphics[width=1\linewidth]{manuscript_files/figure-latex/figconf-1} \caption{Interval coverage for the four sampling-analysis combinations. The rows indicate the proportion of dependent error and the columns indicate the response type. The solid, black line in each plot represents 95\% coverage.}\label{fig:figconf}
\end{figure}

\hypertarget{application}{%
\section{Application}\label{application}}

The Environmental Protection Agency (EPA), states, and tribes
periodically conduct National Aquatic Research Surveys (NARS) in the
United States to assess the water quality of various bodies of water. We
will use the 2012 National Lakes Assessment (NLA), which measures
various aspects of lake health and water quality in lakes in the
contiguous United States, to study mercury concentration. Although we
know the true mean mercury concentration values for the 986 lakes from
the 2012 NLA, we will explore whether or not we obtain an adequately
precise estimate for the realized mean mercury concentration if we
sample only 100 of the 986 lakes.

Figure \ref{fig:figdata} shows that mercury concentration is
right-skewed, with most lakes having a low value of mercury
concentration but a few having a much higher concentration. Mercury
concentration exhibits some spatial correlation, with high mercury
concentrations in lakes in the northeast and north central United
States. The realized mean mercury concentration in the 986 lakes is
103.2 ng / g.

\begin{figure}

{\centering \includegraphics[width=0.49\linewidth]{manuscript_files/figure-latex/figdata-1} \includegraphics[width=0.49\linewidth]{manuscript_files/figure-latex/figdata-2} 

}

\caption{Population distribution of mercury concentration (hg) for 986 lakes in the contiguous United States in a spatial layout (left) and a histogram (right).}\label{fig:figdata}
\end{figure}

\begin{table}[ht]
\centering
\begin{tabular}{lrrrr}
  \hline
Approach & Estimate & SE & 95\% LB & 95\% UB \\ 
  \hline
IRS-Design & 112.7 & 8.8 & 95.4 & 129.9 \\ 
  IRS-Model & 110.5 & 7.9 & 95.0 & 125.9 \\ 
  GRTS-Design & 101.8 & 6.1 & 89.8 & 113.7 \\ 
  GRTS-Model & 102.3 & 5.9 & 90.8 & 113.9 \\ 
   \hline
\end{tabular}
\caption{\label{tab:appliedtab} Application of design-based and model-based approaches to the NLA data set on mercury concentration. The true mean concentration is 103.2 ng / g.} 
\end{table}

We selected a single IRS sample and a single GRTS sample and estimated
the mean mercury concentration and its standard error using using
design-based and model-based approaches; Table \ref{tab:appliedtab}
shows the results. For all four sampling-analysis combinations, the true
realized mean mercury concentration is within the bounds of the 95\%
intervals. However, we should not generalize these results to any other
data or even to other samples from these data. But, we do note a couple
of patterns. The design-based IRS analysis shows the largest standard
error: a likely reason is that this is the only approach that does not
incorporate any spatial information regarding mercury concentration
across the contiguous United States. We also see that both approaches
using the GRTS sample have a lower standard error than the both
approaches using the IRS sample. We would expect this to be the case for
most samples because mercury concentration exhibits spatial patterning,
so a spatially balanced sample should usually yield a lower standard
error.

To better understand the dependence structure in mercury concentration,
the empirical semivariogram and corresponding fit of the model-based
approaches can be visualized. The empirical semivariogram quantifies the
halved squared differences (semivariance) among response values at
different distances apart. If a process exhibits strong spatial
dependence, the empirical semivariogram will have small values at small
distances and large values at large distances. Figure \ref{fig:figsv}
shows the empirical semivariogram for GRTS-Model, displaying the average
semivariance for several distances. Overlain onto Figure \ref{fig:figsv}
is the estimated semivariance obtained using the covariance parmaeters
from the REML fit of GRTS-Model. Figure \ref{fig:figsv} provides
evidence that there is strong correlation in mercury concentration among
the sites.

\begin{figure}

{\centering \includegraphics[width=0.85\linewidth]{manuscript_files/figure-latex/figsv-1} 

}

\caption{The empirical semivariogram (black circles) of mercury concentration against the REML fit using the estimated covariance parameters (black line) from GRTS-Model.}\label{fig:figsv}
\end{figure}

\hypertarget{sec:discussion}{%
\section{Discussion}\label{sec:discussion}}

The design-based and model-based approaches to inference are
fundamentally different paradigms by which samples are selected and data
are analyzed. The design-based approach incorporates randomness through
sampling to estimate a population parameter. The model-based approach
incorporates randomness through distributional assumptions to predict
the realized values of a random process. Though these approaches have
often been compared in the literature both from theoretical and
analytical perspectives, our contribution lies in studying them in a
spatial context while implementing spatially balanced sampling. Aside
from the theoretical differences described, a few analytical findings
from the simulation study are particularly notable. First, the sampling
decision (GRTS vs IRS) is most important when using a design-based
analysis. Though GRTS-Model still outperformed IRS-Model, the
model-based analysis mitigated much of the inefficiency of the IRS
sample. Second, independent of the analysis approach, there is no reason
to use IRS over GRTS for sampling spatial data, as GRTS-Design and
GRTS-Model generally performed at least as well as their IRS
counterparts when there was no spatial correlation and noticeably better
than there IRS counterparts when there was spatial correlation. Third,
The stronger the spatial correlation, the larger the gap in rMS(P)E
between IRS-Design and the other sampling-analysis combinations. Fourth
and finally, interval coverage for the normal response was very close to
95\% for all sample sizes, while interval coverage for the lognormal
response was not very close to 95\% until \(n = 200\).

There are several benefits and drawbacks of the design-based and
model-based approaches for spatial data, some of which we have not yet
discussed but are worthy of consideration in future research.
Design-based approaches are often computationally efficient, while
model-based estimation of covariance parameters can be computationally
burdensome, especially for likelihood-based methods such as REML that
rely on inverting a covariance matrix. The design-based approach also
more naturally handles binary data, free from the more complicated
logistic regression formulation commonly used to handle binary data in a
model-based approach. The model-based approach, however, can more
naturally quantify the relationship between covariates (predictor
variables) and the response variable. The model-based approach also
yields estimated spatial covariance parameters, which help better
understand the process of study. Model selection is also possible using
model-based approaches and criteria such as cross validation, likelihood
ratio tests, or AIC (Akaike, 1974). Model-based approaches are capable
of more efficient small-area estimation than design-based approaches by
leveraging distributional assumptions in areas with few observed sites.
Model-based approaches can also compute site-by-site predictions at
unobserved locations and use them to construct informative
visualizations. The benefits and drawbacks of both approaches, alongside
our theoretical and analytical comparisons, should be seriously
considered when choosing among them. This is especially true from an
analysis perspective, as we found that using a spatially balanced
sampling algorithm benefits both design-based and model-based analyses.

\hypertarget{data-and-code-availability}{%
\section*{Data and Code Availability}\label{data-and-code-availability}}
\addcontentsline{toc}{section}{Data and Code Availability}

This manuscript has a supplementary R package that contains all of the
data and code used. Instructions for download at available at
\url{https://github.com/michaeldumelle/DvMsp}.

\hypertarget{supplementary-material}{%
\section*{Supplementary Material}\label{supplementary-material}}
\addcontentsline{toc}{section}{Supplementary Material}

In the supplementary material, we provide tables presenting summary
statistics for all 36 simulation scenarios.

\hypertarget{acknowledgements}{%
\section*{Acknowledgements}\label{acknowledgements}}
\addcontentsline{toc}{section}{Acknowledgements}

The views expressed in this manuscript are those of the authors and do
not necessarily represent the views or policies of the U.S.
Environmental Protection Agency. Any mention of trade names, products,
or services does not imply an endorsement by the U.S. government or the
U.S. Environmental Protection Agency. The U.S. Environmental Protection
Agency does not endorse any commercial products, services, or
enterprises.

\hypertarget{references}{%
\section*{References}\label{references}}
\addcontentsline{toc}{section}{References}

\hypertarget{refs}{}
\leavevmode\hypertarget{ref-akaike1974new}{}%
Akaike, H., 1974. A new look at the statistical model identification.
IEEE Transactions on Automatic Control 19, 716--723.

\leavevmode\hypertarget{ref-barabesi2011sampling}{}%
Barabesi, L., Franceschi, S., 2011. Sampling properties of spatial total
estimators under tessellation stratified designs. Environmetrics 22,
271--278.

\leavevmode\hypertarget{ref-benedetti2017spatially}{}%
Benedetti, R., Piersimoni, F., 2017. A spatially balanced design with
probability function proportional to the within sample distance.
Biometrical Journal 59, 1067--1084.

\leavevmode\hypertarget{ref-benedetti2017spatiallyreview}{}%
Benedetti, R., Piersimoni, F., Postiglione, P., 2017. Spatially balanced
sampling: A review and a reappraisal. International Statistical Review
85, 439--454.

\leavevmode\hypertarget{ref-breiman2001random}{}%
Breiman, L., 2001. Random forests. Machine Learning 45, 5--32.

\leavevmode\hypertarget{ref-brus1997random}{}%
Brus, D., De Gruijter, J., 1997. Random sampling or geostatistical
modelling? Choosing between design-based and model-dased sampling
strategies for soil (with discussion). Geoderma 80, 1--44.

\leavevmode\hypertarget{ref-brus2021statistical}{}%
Brus, D.J., 2021. Statistical approaches for spatial sample survey:
Persistent misconceptions and new developments. European Journal of Soil
Science 72, 686--703.

\leavevmode\hypertarget{ref-chan2020bayesian}{}%
Chan-Golston, A.M., Banerjee, S., Handcock, M.S., 2020. Bayesian
inference for finite populations under spatial process settings.
Environmetrics 31, e2606.

\leavevmode\hypertarget{ref-chiles1999geostatistics}{}%
Chiles, J.-P., Delfiner, P., 1999. Geostatistics: Modeling Spatial
Uncertainty. John Wiley \& Sons, New York.

\leavevmode\hypertarget{ref-cicchitelli2012model}{}%
Cicchitelli, G., Montanari, G.E., 2012. Model-assisted estimation of a
spatial population mean. International Statistical Review 80, 111--126.

\leavevmode\hypertarget{ref-cooper2006sampling}{}%
Cooper, C., 2006. Sampling and variance estimation on continuous
domains. Environmetrics 17, 539--553.

\leavevmode\hypertarget{ref-cressie1993statistics}{}%
Cressie, N., 1993. Statistics for spatial data. John Wiley \& Sons.

\leavevmode\hypertarget{ref-de1990model}{}%
De Gruijter, J., Ter Braak, C., 1990. Model-free estimation from spatial
samples: A reappraisal of classical sampling theory. Mathematical
Geology 22, 407--415.

\leavevmode\hypertarget{ref-diggle2010geostatistical}{}%
Diggle, P.J., Menezes, R., Su, T.-l., 2010. Geostatistical inference
under preferential sampling. Journal of the Royal Statistical Society:
Series C (Applied Statistics) 59, 191--232.

\leavevmode\hypertarget{ref-dumelle2021spsurvey}{}%
Dumelle, M., Kincaid, T.M., Olsen, A.R., Weber, M.H., 2021. Spsurvey:
Spatial sampling design and analysis.

\leavevmode\hypertarget{ref-fix1989discriminatory}{}%
Fix, E., Hodges, J.L., 1989. Discriminatory analysis. Nonparametric
discrimination: Consistency properties. International Statistical
Review/Revue Internationale de Statistique 57, 238--247.

\leavevmode\hypertarget{ref-foster2017spatially}{}%
Foster, S.D., Hosack, G.R., Lawrence, E., Przeslawski, R., Hedge, P.,
Caley, M.J., Barrett, N.S., Williams, A., Li, J., Lynch, T., others,
2017. Spatially balanced designs that incorporate legacy sites. Methods
in Ecology and Evolution 8, 1433--1442.

\leavevmode\hypertarget{ref-grafstrom2012spatiallypoisson}{}%
Grafström, A., 2012. Spatially correlated poisson sampling. Journal of
Statistical Planning and Inference 142, 139--147.

\leavevmode\hypertarget{ref-grafstrom2013well}{}%
Grafström, A., Lundström, N.L., 2013. Why well spread probability
samples are balanced. Open Journal of Statistics 3, 36--41.

\leavevmode\hypertarget{ref-grafstrom2012spatially}{}%
Grafström, A., Lundström, N.L., Schelin, L., 2012. Spatially balanced
sampling through the pivotal method. Biometrics 68, 514--520.

\leavevmode\hypertarget{ref-grafstrom2018spatially}{}%
Grafström, A., Matei, A., 2018. Spatially balanced sampling of
continuous populations. Scandinavian Journal of Statistics 45, 792--805.

\leavevmode\hypertarget{ref-hansen1983evaluation}{}%
Hansen, M.H., Madow, W.G., Tepping, B.J., 1983. An evaluation of
model-dependent and probability-sampling inferences in sample surveys.
Journal of the American Statistical Association 78, 776--793.

\leavevmode\hypertarget{ref-harville1977maximum}{}%
Harville, D.A., 1977. Maximum likelihood approaches to variance
component estimation and to related problems. Journal of the American
Statistical Association 72, 320--338.

\leavevmode\hypertarget{ref-higham2021sptotal}{}%
Higham, M., Ver Hoef, J., Frank, B., Dumelle, M., 2021. Sptotal:
Predicting totals and weighted sums from spatial data.

\leavevmode\hypertarget{ref-horvitz1952generalization}{}%
Horvitz, D.G., Thompson, D.J., 1952. A generalization of sampling
without replacement from a finite universe. Journal of the American
Statistical Association 47, 663--685.

\leavevmode\hypertarget{ref-lohr2009sampling}{}%
Lohr, S.L., 2009. Sampling: Design and analysis. Nelson Education.

\leavevmode\hypertarget{ref-patterson1971recovery}{}%
Patterson, H.D., Thompson, R., 1971. Recovery of inter-block information
when block sizes are unequal. Biometrika 58, 545--554.

\leavevmode\hypertarget{ref-robertson2013bas}{}%
Robertson, B., Brown, J., McDonald, T., Jaksons, P., 2013. BAS: Balanced
acceptance sampling of natural resources. Biometrics 69, 776--784.

\leavevmode\hypertarget{ref-robertson2018halton}{}%
Robertson, B., McDonald, T., Price, C., Brown, J., 2018. Halton
iterative partitioning: Spatially balanced sampling via partitioning.
Environmental and Ecological Statistics 25, 305--323.

\leavevmode\hypertarget{ref-sarndal2003model}{}%
Särndal, C.-E., Swensson, B., Wretman, J., 2003. Model assisted survey
sampling. Springer Science \& Business Media.

\leavevmode\hypertarget{ref-schabenberger2017statistical}{}%
Schabenberger, O., Gotway, C.A., 2017. Statistical methods for spatial
data analysis. CRC press.

\leavevmode\hypertarget{ref-sen1953estimate}{}%
Sen, A.R., 1953. On the estimate of the variance in sampling with
varying probabilities. Journal of the Indian Society of Agricultural
Statistics 5, 127.

\leavevmode\hypertarget{ref-sterba2009alternative}{}%
Sterba, S.K., 2009. Alternative model-based and design-based frameworks
for inference from samples to populations: From polarization to
integration. Multivariate Behavioral Research 44, 711--740.

\leavevmode\hypertarget{ref-stevens2003variance}{}%
Stevens, D.L., Olsen, A.R., 2003. Variance estimation for spatially
balanced samples of environmental resources. Environmetrics 14,
593--610.

\leavevmode\hypertarget{ref-stevens2004spatially}{}%
Stevens, D.L., Olsen, A.R., 2004. Spatially balanced sampling of natural
resources. Journal of the American Statistical Association 99, 262--278.

\leavevmode\hypertarget{ref-verhoef2002sampling}{}%
Ver Hoef, J., 2002. Sampling and geostatistics for spatial data.
Ecoscience 9, 152--161.

\leavevmode\hypertarget{ref-verhoef2008spatial}{}%
Ver Hoef, J.M., 2008. Spatial methods for plot-based sampling of
wildlife populations. Environmental and Ecological Statistics 15, 3--13.

\leavevmode\hypertarget{ref-ver2013comparison}{}%
Ver Hoef, J.M., Temesgen, H., 2013. A comparison of the spatial linear
model to nearest neighbor (k-nn) methods for forestry applications. PlOS
ONE 8, e59129.

\leavevmode\hypertarget{ref-wang2013design}{}%
Wang, J.-F., Jiang, C.-S., Hu, M.-G., Cao, Z.-D., Guo, Y.-S., Li, L.-F.,
Liu, T.-J., Meng, B., 2013. Design-based spatial sampling: Theory and
implementation. Environmental Modelling \& Software 40, 280--288.

\leavevmode\hypertarget{ref-wang2012review}{}%
Wang, J.-F., Stein, A., Gao, B.-B., Ge, Y., 2012. A review of spatial
sampling. Spatial Statistics 2, 1--14.

\leavevmode\hypertarget{ref-wolfinger1994computing}{}%
Wolfinger, R., Tobias, R., Sall, J., 1994. Computing gaussian
likelihoods and their derivatives for general linear mixed models. SIAM
Journal on Scientific Computing 15, 1294--1310.


\end{document}


