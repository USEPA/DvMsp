\documentclass[]{elsarticle} %review=doublespace preprint=single 5p=2 column
%%% Begin My package additions %%%%%%%%%%%%%%%%%%%
\usepackage[hyphens]{url}

  \journal{An awesome journal} % Sets Journal name


\usepackage{lineno} % add
  \linenumbers % turns line numbering on
\providecommand{\tightlist}{%
  \setlength{\itemsep}{0pt}\setlength{\parskip}{0pt}}

\usepackage{graphicx}
\usepackage{booktabs} % book-quality tables
%%%%%%%%%%%%%%%% end my additions to header

\usepackage[T1]{fontenc}
\usepackage{lmodern}
\usepackage{amssymb,amsmath}
\usepackage{ifxetex,ifluatex}
\usepackage{fixltx2e} % provides \textsubscript
% use upquote if available, for straight quotes in verbatim environments
\IfFileExists{upquote.sty}{\usepackage{upquote}}{}
\ifnum 0\ifxetex 1\fi\ifluatex 1\fi=0 % if pdftex
  \usepackage[utf8]{inputenc}
\else % if luatex or xelatex
  \usepackage{fontspec}
  \ifxetex
    \usepackage{xltxtra,xunicode}
  \fi
  \defaultfontfeatures{Mapping=tex-text,Scale=MatchLowercase}
  \newcommand{\euro}{€}
\fi
% use microtype if available
\IfFileExists{microtype.sty}{\usepackage{microtype}}{}
\bibliographystyle{elsarticle-harv}
\ifxetex
  \usepackage[setpagesize=false, % page size defined by xetex
              unicode=false, % unicode breaks when used with xetex
              xetex]{hyperref}
\else
  \usepackage[unicode=true]{hyperref}
\fi
\hypersetup{breaklinks=true,
            bookmarks=true,
            pdfauthor={},
            pdftitle={A comparison of design-based and model-based approaches for finite population spatial data.},
            colorlinks=false,
            urlcolor=blue,
            linkcolor=magenta,
            pdfborder={0 0 0}}
\urlstyle{same}  % don't use monospace font for urls

\setcounter{secnumdepth}{5}
% Pandoc toggle for numbering sections (defaults to be off)

% Pandoc citation processing
\newlength{\cslhangindent}
\setlength{\cslhangindent}{1.5em}
\newlength{\csllabelwidth}
\setlength{\csllabelwidth}{3em}
% for Pandoc 2.8 to 2.10.1
\newenvironment{cslreferences}%
  {}%
  {\par}
% For Pandoc 2.11+
\newenvironment{CSLReferences}[2] % #1 hanging-ident, #2 entry spacing
 {% don't indent paragraphs
  \setlength{\parindent}{0pt}
  % turn on hanging indent if param 1 is 1
  \ifodd #1 \everypar{\setlength{\hangindent}{\cslhangindent}}\ignorespaces\fi
  % set entry spacing
  \ifnum #2 > 0
  \setlength{\parskip}{#2\baselineskip}
  \fi
 }%
 {}
\usepackage{calc}
\newcommand{\CSLBlock}[1]{#1\hfill\break}
\newcommand{\CSLLeftMargin}[1]{\parbox[t]{\csllabelwidth}{#1}}
\newcommand{\CSLRightInline}[1]{\parbox[t]{\linewidth - \csllabelwidth}{#1}\break}
\newcommand{\CSLIndent}[1]{\hspace{\cslhangindent}#1}

% Pandoc header

\usepackage{bm} \usepackage{bbm} \usepackage{color} \DeclareMathOperator{\var}{{var}} \DeclareMathOperator{\cov}{{cov}}

\begin{document}
\begin{frontmatter}

  \title{A comparison of design-based and model-based approaches for
finite population spatial data.}
    \author[USEPA]{Michael Dumelle\corref{1}}
   \ead{Dumelle.Michael@epa.gov} 
    \author[STLAW]{Matt Higham}
   \ead{mhigham@stlaw.edu} 
    \author[OSU]{Lisa Madsen}
  
    \author[USEPA]{Anthony R. Olsen}
  
    \author[NOAA]{Jay M. Ver Hoef}
  
      \address[USEPA]{United States Environmental Protection Agency, 200
SW 35th St, Corvallis, Oregon, 97333}
    \address[STLAW]{Saint Lawrence University Department of Mathematics,
Computer Science, and Statistics, 23 Romoda Drive, Canton, New York,
13617}
    \address[OSU]{Oregon State University Department of Statistics, 239
Weniger Hall, Corvallis, Oregon, 97331}
    \address[NOAA]{Marine Mammal Laboratory, Alaska Fisheries Science
Center, National Oceanic and Atmospheric Administration, Seattle,
Washington, 98115}
      \cortext[1]{Corresponding Author}
  
  \begin{abstract}
  This is the abstract.
  \end{abstract}
  
 \end{frontmatter}

\emph{Text based on elsarticle sample manuscript, see
\url{http://www.elsevier.com/author-schemas/latex-instructions\#elsarticle}}

Potential Journals:

\begin{itemize}
\tightlist
\item
  Ecological Applications
\item
  Methods in Ecology and Evolution
\item
  Journal of Applied Ecology
\item
  Environmetrics
\item
  Environmental and Ecological Statistics
\end{itemize}

\hypertarget{sec:introduction}{%
\section{Introduction}\label{sec:introduction}}

There are two general approaches for using data to make statistical
inferences about a population: design-based and model-based. When data
cannot be obtained for all units in a population (population units),
data on a subset of the population units is collected and called a
sample. In the design-based approach, inferences about the underlying
population are informed from a probabilistic process in which population
units are selected to be in the sample. Alternatively, in the
model-based approach, inferences are made from specific assumptions
about the underlying process that generated the data. Each paradigm has
a deep historical context (Sterba, 2009) and its own set of general
advantages (Hansen et al., 1983).

Though the design-based and model-based approaches apply to statistical
inference in a broad sense, we focus on comparing these approaches for
spatial data. We define spatial data as data that incorporates the
specific locations of the population units into either the design or
estimation process. De Gruijter and Ter Braak (1990) give an early
comparison of design-based and model-based approaches for spatial data,
quashing the belief that design-based approaches could not be used for
spatially correlated data. Thereafter, several comparisons between
design-based and model-based for spatial data have been considered (Brus
and De Gruijter, 1997; Ver Hoef, 2002; Ver Hoef, 2008). Cooper (2006)
review the two approaches in an ecological context before introducing a
``model-assisted'' variance estimator that combines aspects from each
approach. In addition to Cooper (2006), there has been substantial
research and development into estimators that use both design and
model-based principles (see e.g. Cicchitelli and Montanari (2012),
Chan-Golston et al. (2020) for a Bayesian approach, and Sterba (2009)).
More recent overviews include Brus (2020) and Wang et al. (2012).

Though comparisons between design-based and model-based approaches to
spatial data have been studied, no numerical comparison has been made
between design-based approaches that incorporate spatial locations and
model-based approaches. In this manuscript, we compare design-based
approaches that incorporate spatial locations to model-based approaches
for spatial data. Though these comparisons generalize to both finite
populations (e.g.~point resources) and infinite populations (e.g.~linear
and areal resources), we focus on applications to finite populations.
The rest of the manuscript is organized as follows. In Section
\ref{sec:background}, we compare sampling and estimation procedures
between the design-based approach and the model-based approach for
spatial data. In Section \ref{sec:numstudy}, we use a simulation
approach to study the behavior and performance of both approaches. In
Section \ref{application}, we use both approaches to analyze real data.
And in Section \ref{sec:discussion}, we end with a discussion and
provide directions for future research.

\hypertarget{sec:background}{%
\section{Background}\label{sec:background}}

The design-based and model-based approaches incorporate randomness in
fundamentally different ways. In this section, we describe the role of
randomness and its effects on subsequent inferences. We then discuss
specific inference methods for the design-based and model-based
approaches for spatial data.

\hypertarget{subsec:dvm_compare}{%
\subsection{Comparing Design-Based and Model-Based
Approaches}\label{subsec:dvm_compare}}

The design-based approach assumes the population is fixed. Randomness is
incorporated via the selection of units in a sampling frame according to
a sampling design. A sampling frame is the set of all units available to
be sampled. A sampling design assigns a positive probability of
inclusion (inclusion probability) to each unit in the sampling frame.
Some examples of commonly used sampling designs include simple random
sampling, stratified random sample, and cluster sampling. These sampling
designs tend to select units from the sampling frame independently of
other units, so we call them ``Independent Random Sampling'' (IRS)
designs. Sampling designs incorporating the spatial locations of units
in the sample frame are called spatially balanced designs. Spatially
balanced designs can be obtained using the Generalized Random
Tessellation Stratified (GRTS) algorithm (Stevens and Olsen, 2004),
which we discuss in more detail in Section \ref{subsec:spb_design}. The
design-based approach combines the randomness of the sampling design and
the data collected via the sample to estimate parameters (e.g.~means and
totals) of a population. Generally, these population parameters are
assumed to be fixed, unknown constants.

Treating the data as fixed and incorporating randomness through the
sampling design yields estimators having very few other assumptions.
Confidence intervals for these types of estimators are typically derived
using limiting arguments that incorporate all possible randomizations of
sampling units selected via the sampling design. Means and totals, for
example, are asymptotically normally distributed (normal) by the Central
Limit Theorem (under some assumptions). If we repeatedly sample the
surface, then 95\% of all 95\% confidence intervals constructed from a
procedure with appropriate coverage will contain the true, fixed mean.
Särndal et al. (2003) and Lohr (2009) provide thorough reviews of the
design-based approach.

The model-based approach assumes the data are a random realization of a
data-generating process. Randomness is incorporated through
distributional assumptions on this process. Strictly speaking,
randomness need not be incorporated through random sampling, though
Diggle et al. (2010) warn against preferential sampling Preferential
sampling occurs when the process generating the data locations and the
process being modeled are not independent of one another. To guard
against preferential sampling, model-based approaches often still
implement random sampling.

Instead of estimating fixed but unknown parameters (as in the
design-based approach), the goal of model-based inference in the spatial
context is often to predict a realized variable, or value. For example,
suppose the realized mean of all population units is the value of
interest. Instead of \emph{estimating} a fixed, unknown mean, we are
\emph{predicting} the value of the mean, a random variable. Prediction
intervals are then derived leveraging assumptions of the data generating
process. If we repeatedly generate the response values from a fixed
spatial process and obtained a sample, then 95\% of all 95\% prediction
intervals constructed from a procedure with appropriate coverage will
contain their respective realized means. Cressie (1993) and
Schabenberger and Gotway (2017) provide reviews of model-based
approaches for spatial data. A visual comparison of the design-based and
model-based assumptions is provided in Figure \ref{fig:fig1} (Brus
(2020) provides a similar figure).

\begin{figure}
\includegraphics[width=1\linewidth]{manuscript_files/figure-latex/fig1-1} \caption{A comparison of sampling under the design-based and model-based frameworks. Points circled are those that are sampled. In the top row, we have one fixed population, and three random samples of n = 4. The response values at each site are fixed, but we obtain different estimates for the mean response because the randomly sampled sites vary from sample to sample. In the bottom row, we have three realizations of the same spatial process sampled at the same locations. The spatial process generating the response values has a single mean, but the realized mean is different in each of the three panels.}\label{fig:fig1}
\end{figure}

\hypertarget{subsec:spb_design}{%
\subsection{Spatially Balanced Design and
Analysis}\label{subsec:spb_design}}

Spatially balanced samples can be obtained using the design-based
approach. Spatially balanced samples are useful because parameter
estimates from these samples tend to vary less than parameter estimates
from samples that are not spatially balanced (Barabesi and Franceschi,
2011; Benedetti et al., 2017; Grafström and Lundström, 2013; Robertson
et al., 2013; Stevens and Olsen, 2004; Wang et al., 2013). The first
spatially balanced sampling algorithm that saw widespread use was the
Generalized Random Tessellation Stratified (GRTS) algorithm (Stevens and
Olsen, 2004). To quantify the spatial balance of a sample, Stevens and
Olsen (2004) proposed loss metrics based on Voroni polygons. Since GRTS
was developed, several other spatially balanced sampling algorithms have
emerged, including the Local Pivotal Method (Grafström et al., 2012;
Grafström and Matei, 2018), Spatially Correlated Poisson Sampling
(Grafström, 2012), Balanced Acceptance Sampling (Robertson et al.,
2013), Within-Sample-Distance Sampling (Benedetti and Piersimoni, 2017),
and Halton Iterative Partitioning Sampling (Robertson et al., 2018). In
this manuscript, we use Generalized Random Tessellation Stratified
(GRTS) sampling because it has several attractive properties: GRTS
sampling accommodates finite and infinite sampling frames; accommodates
equal, unequal, and proportional (to) size inclusion probabilities;
accommodates legacy (historical) sampling; accommodates a minimum
distance between units in a sample; accommodates reverse hierarchically
ordered replacement units in a sample (replacement units are units
available to be sampled if an original unit cannot be sampled); and is
available in the spsurvey R package Dumelle et al. (2021).

The GRTS algorithm samples from finite and infinite populations by
utilizing a mapping between two-dimensional and one-dimensional space.
The units in the two-dimensional sampling frame are divided into cells
using a hierarchical address. This hierarchical address is then used to
map the units from two-dimensional space to a one-dimensional line where
each unit's line length equals its inclusion probability. A systematic
sample is conducted on the line and linked back to a unit in
two-dimensional space, which results in the desired sample. Stevens and
Olsen (2004) and Dumelle et al. (2021) provide further details.

After selecting a GRTS sample, data are collected and used to estimate
population parameters. To unbiasedly estimate population means and
totals from sample data, one can use the Horvitz-Thompson estimator
(Horvitz and Thompson, 1952). If \(\tau\) is a population total, the
Horvitz-Thompson estimate of \(\tau\), denoted by \(\hat{\tau}_{ht}\),
is is given by \begin{align}\label{eq:ht}
  \hat{\tau}_{ht} = \sum_{i = 1}^n Z_i \pi_i^{-1},
\end{align} where \(Z_i\) is the value of the \(i\)th unit in the sample
and \(\pi_i\) is the inclusion probability of the \(i\)th unit in the
sample. An estimate of the population mean is obtained by dividing
\(\hat{\tau}_{ht}\) by the population size.

While the Horvitz-Thompson estimator is unbiased for population means
and totals, it is also important to quantify the uncertainty in these
estimates. Horvitz and Thompson (1952) and Sen (1953) provide variance
estimators for \(\hat{\tau}_{ht}\), but they have two drawbacks. First,
these estimators rely on calculating \(\pi_{ij}\), the probability that
unit \(i\) and unit \(j\) are both in the sample -- this quantity can be
challenging if not impossible to calculate analytically. Second, these
estimators ignore the spatial locations of the units in the sampling
frame. To address these two drawbacks simultaneously, Stevens and Olsen
(2003) proposed the local neighborhood variance estimator. The local
neighborhood variance estimator does not rely on \(\pi_{ij}\) and
incorporates spatial locations -- for technical details see Stevens and
Olsen (2003). Stevens and Olsen (2003) show the local neighborhood
variance estimator tends reduce \(\text{Var}(\hat{\tau})\) compared to
variance estimators ignoring spatial locations, yielding narrower
confidence intervals for \(\tau\).

\hypertarget{finite-population-block-kriging}{%
\subsection{Finite Population Block
Kriging}\label{finite-population-block-kriging}}

Finite Population Block Kriging (FPBK) is a model-based approach that
expands the geostatistical Kriging framework to the finite population
setting (Ver Hoef, 2008). Instead of basing inference off of a specific
sampling design, we assume the data are generated by a spatial process.
Ver Hoef (2008) gives details on the theory of FPBK, but some of the
basic principles are summarized below. Let
\({\mathbf{z} \equiv \{\text{z}(s_1), \text{z}(s_2), . . . , \text{z}(s_N) \}}\)
be a response vector at locations \(s_1\), \(s_2\), . . . , \(s_N\) that
can be measured at the \(N\) population units and is represented as an
\(N \times 1\) vector. Suppose we want to predict some linear function
of the response variable,
\(f(\mathbf{z}) = \mathbf{b}^\prime \mathbf{z}\), where
\(\mathbf{b}^\prime\) is a \(1 \times N\) vector of weights. For
example, if we want to predict the population total across all
population units, then we would use a vector of 1's for the weights.

We often only have a sample of the \(N\) population units. Denoting
quantities that are part of the sampled population units with a
subscript \emph{s} and quantities that are part of the unsampled
population units with a subscript \emph{u},

\begin{equation}
\begin{pmatrix} \label{equation:Zmarginal}
    \mathbf{z}_s      \\
    \mathbf{z}_u
\end{pmatrix}
=
\begin{pmatrix}
  \mathbf{X}_s    \\
  \mathbf{X}_u
\end{pmatrix}
\bm{\beta} +
\begin{pmatrix}
\bm{\delta}_s    \\
\bm{\delta}_u
\end{pmatrix},
\end{equation} where \(\mathbf{X}_s\) and \(\mathbf{X}_u\) are the
design matrices for the sampled and unsampled population units,
respectively; \(\beta\) is the parameter vector of fixed effects; and
\(\bm{\delta}_s\) and \(\bm{\delta}_u\) are random errors for the
sampled and unsampled population units, respectively. Denoting
\(\bm{\delta} \equiv [\bm{\delta}_s \,\, \bm{\delta}_u]'\), we assume
the expectation of \(\bm{\delta}\) equals \(\mathbf{0}\).

In addition to assuming the expectation of \(\bm{\delta}\) equals
\(\mathbf{0}\), we also assume that there is spatial correlation in
\(\bm{\delta}\) that can be modeled using a covariance function. It is
common to assume the covariance function is second-order stationary and
isotropic (Cressie, 1993), and that the spatial covariance decreases as
the separation between population units increases. Many spatial
covariance functions exist, but the primary function we use throughout
the simulations and applications in this manuscript is the exponential
covariance function: the \(i,j\)th entry for
\(\mathop{\mathrm{{cov}}}(\bm{\delta})\) is \mbox{}
\begin{align}\label{equation:expcov}
\mathop{\mathrm{{cov}}}(\delta_i, \delta_j) = 
\begin{cases} 
\sigma^2_{ps}\exp(-h_{i,j}/\phi) & h_{i,j} > 0 \\
\sigma^2_{ps} + \sigma^2_n & h_{i,j} = 0
\end{cases}
,
\end{align} where \(\sigma^2_{ps}\) is the partial sill measuring
coarse-scale (correlated) variability, \(\sigma^2_{n}\) is the nugget
measuring fine-scale (independent) variability, \(\phi\) is the range
parameter measuring the distance-decay rate of the covariance, and
\(h_{i,j}\) is the Euclidean distance between population units \(i\) and
\(j\). Any spatial covariance function could be used in the place of the
exponential, however, including functions that allow for
non-stationarity or anisotropy (Chiles and Delfiner, 1999, pp. 80--93).

With the above model formulation, the Best Linear Unbiased Predictor
(BLUP) for \(f(\mathbf{b}'\mathbf{z})\) and its prediction variance can
be computed. While details of the derivation are in (Ver Hoef, 2008), we
note here that the predictor and its variance are both moment-based,
meaning they don't rely on any distributional assumptions.

We note that we only use FPBK in this paper in order to focus more on
comparing the design-based and model-based approaches. However,
k-nearest-neighbors (Fix and Hodges, 1951; Ver Hoef and Temesgen, 2013),
random forest (Breiman, 2001), Bayesian models (Chan-Golston et al.,
2020), among others, can also be used to obtain predictions for a mean
or total from spatially correlated responses of a finite population. We
choose to use FPBK because it is faster than a Bayesian approach and
random forest and because Ver Hoef and Temesgen (2013) showed that the
method outperforms k-nearest-neighbors in many scenarios.

\hypertarget{sec:numstudy}{%
\section{Numerical Study}\label{sec:numstudy}}

We used a numerical simulation study to investigate performance of four
design-analysis combinations, summarized in Table
\ref{tab:designanalysis}.

\begin{table}[ht]
\centering
\begin{tabular}{r|ll}
  \hline
 & Design & Model \\ 
  \hline
IRS & IRS-Design & IRS-Model \\ 
  GRTS & GRTS-Design & GRTS-Model \\ 
   \hline
\end{tabular}
\caption{\label{tab:designanalysis} Types of Sampling Design and Analysis combinations considered in the simulation study. The rows give the two types of sampling designs while the columns give the two types of analyses.} 
\end{table}

We used a crossed design with the simulation parameters given in Table
\ref{tab:parmtab} for a total of 36 scenarios. All scenarios used
exponential correlation with an effective range of \(\sqrt{2}\) for
\(N = 900\) response values simulated on the unit square in either
random locations (Layout = Random) or gridded locations (Layout =
Gridded). The mean for the spatial process generating the response was
set to zero.

For the lognormal scenarios, the response values were simulated using
the specified correlation parameters using a normal distribution and
were subsequently exponentiated. A total variance of 2 and a mean of 0
on the normal scale is equivalent to a total variance of 47 and a mean
of 2.72 after exponentiation. Therefore, when the model-based methods
were used for lognormal response, the correlation was mis-specified. We
chose to simulate values with a lognormal distribution so that we could
test the model-based analysis approach with a mis-specified model and so
that we could test both analysis approaches on data that exhibits a
large amount of skewness.

\begin{table}[ht]
\centering
\begin{tabular}{r|lll}
   \hline
Sample Size (n) & 50 & 100 & 200 \\ 
  Layout & Random & Gridded & - \\ 
  Proportion of Dependent Error & 0 & 0.5 & 0.9 \\ 
  Response Type & Normal & Lognormal & - \\ 
   \hline
\end{tabular}
\caption{\label{tab:parmtab} Simulation parameters. Total variability for all scenarios was 2 so that the partial sill was 0, 1, or 1.8.} 
\end{table}

There were 2000 simulation trials for each of the 36 parameter
combinations. In each trial, response values were generated from a
spatial process with the specified parameters, and a GRTS sample and an
IRS sample were selected. For the GRTS sample, the design-based approach
using the local neighborhood variance (GRTS-Design) and a model-based
approach were applied (GRTS-Model). For the IRS sample, the design-based
approach using the simple random sample variance (IRS-Design) and a
model-based approach were applied (IRS-Model).

The GRTS algorithm and the local neighborhood variance estimator are
available in the \textbf{\textsf{R}} package \texttt{spsurvey} (Dumelle
et al., 2021). FPBK can be readily performed in \texttt{R} with the
\texttt{sptotal} package (Higham et al., 2021). We use \texttt{sptotal}
for both the simulation analysis and the application, estimating
parameters with Restricted Maximum Likelihood (REML).

\textbf{Mike} For design-based, it's really RMSE -- how should we
address this? Figure \ref{fig:figeff} shows the relative efficiency of
the four approaches from Table \ref{tab:designanalysis} with
``IRS-Design'' as the baseline: \mbox{} \begin{equation*}
\text{EFF} = \frac{\text{rMSPE of approach}}{\text{rMSPE of IRS-Design}},
\end{equation*}

where rMSPE is the root-mean-squared-prediction error. When there is no
spatial correlation (top row), the four approaches have approximately
equal rMSPE, even when the assumptions of the model-based approaches are
violated. So, using GRTS or using a spatial model does not result in
much, if any, loss in efficiency even if the response variable is not
spatially correlated. When there is high spatial correlation (bottom
row), the GRTS-Model approach tends to perform best, but difference in
relative efficiency between GRTS-Model and GRTS-Design is small. In the
lognormal, high partial sill settings (bottom-right facet), GRTS-Design
outperforms IRS-Model by a large margin, suggesting that the design
decision (whether to use IRS or GRTS) is more important than the
analysis decision (whether to analyze using model assumptions or not).

Unsurprisingly, Figure \ref{fig:figeff} also shows that, when the
assumptions for GRTS-Model are satisfied, the approach outperforms
GRTS-Design. However, even when the model that generates the data is
different than the model used to fit the data, as in the lognormal
response, the model-based approach still outperforms the design-based
approach when there is a high amount of spatial correlation.

\begin{figure}
\includegraphics[width=1\linewidth]{manuscript_files/figure-latex/figeff-1} \caption{Relative Efficiency of the four design-analysis approaches. The plot is faceted by the type of response on the columns and the partial-sill to total-variance ratio on the rows.}\label{fig:figeff}
\end{figure}

We also studied 95\% interval coverage among the approaches. The
design-based 95\% confidence intervals and model-based 95\% prediction
intervals are constructed using the normal distribution. Justification
for the design-based intervals lies in the asymptotic normality of
totals via the Central Limit Theorem, and justification for the
model-based intervals lies in the normality assumption of the errors.
Figure \ref{fig:figconf} shows the 95\% interval coverage for each of
the four approaches. All four approaches have somewhat similar interval
coverage in all settings, with GRTS-Design having slightly lower
coverage when the response is normal.

In the normal response settings, all approaches have coverage around
95\%. This is expected, as the intervals are also based on the normal
distribution In the lognormal response settings, however, all approaches
have coverage below 95\%. This is also expected, as the intervals are
still based on the normal distribution. In the lognormal response
settings, interval coverage increases both as the sample size increases
and as the strength of spatial dependence increases. This suggests that
the larger the sample size and the stronger the spatial dependence, the
more resistant these intervals are to departures from normality of the
data.

\begin{figure}
\includegraphics[width=1\linewidth]{manuscript_files/figure-latex/figconf-1} \caption{Coverage of the four design-analysis approaches. All confidence intervals are normal-based and have a nominal confidence level of 0.95, marked with a horizontal line. The plot is faceted by the type of response on the columns and the partial-sill to total-variance ratio on the rows.}\label{fig:figconf}
\end{figure}

\hypertarget{application}{%
\section{Application}\label{application}}

The Environmental Protection Agency (EPA), states, and tribes
periodically conduct National Aquatic Research Surveys (NARS) in the
United States to assess the water quality of various bodies of water. We
will use the 2012 National Lakes Assessment (NLA), which measures
various aspects of lake health and quality in lakes in the contiguous
United States, to obtain an interval for mean mercury concentration.
Although we know the true mean mercury concentration values for the 986
lakes from the 2012 NLA, we will explore whether or not we obtain an
adequately precise estimate for the realized mean mercury concentration
if we sample only 100 of the 986 lakes.

\begin{figure}
\includegraphics[width=1\linewidth]{manuscript_files/figure-latex/figdata-1} \caption{Population distribution of mercury concentration for 986 lakes in the contiguous United States.}\label{fig:figdata}
\end{figure}

Figure \ref{fig:figdata} shows that mercury concentration is
right-skewed, with most lakes having a low value of mercury
concentration but a few having a much higher concentration. Mercury
concentration exhibits some spatial correlation, with high mercury
concentrations in lakes in the northeast and north central United
States. The realized mean mercury concentration in the 986 lakes is
103.2 ng / g.

\begin{table}[ht]
\centering
\begin{tabular}{lrrrr}
  \hline
Approach & Estimate & SE & 95\% LB & 95\% UB \\ 
  \hline
IRS-Design & 112.7 & 8.8 & 95.4 & 129.9 \\ 
  IRS-Model & 110.5 & 7.9 & 95.0 & 125.9 \\ 
  GRTS-Design & 101.8 & 6.1 & 89.8 & 113.7 \\ 
  GRTS-Model & 102.3 & 5.9 & 90.8 & 113.9 \\ 
   \hline
\end{tabular}
\caption{\label{tab:appliedtab} Application of design-based and model-based approaches to the NLA data set on mercury concentration. The true mean concentration is 103.2 ng / g.} 
\end{table}

Table \ref{tab:appliedtab} shows the application of a design-based
analysis of an IRS sample, a model-based analysis of an IRS sample, a
design-based analysis of a GRTS sample, and a model-based analysis of a
GRTS sample. For all four analyses, the true realized mean mercury
concentration is within the bounds of the 95\% intervals. However, we
should not generalize the results of this particular realization to any
other data set or even to other potential samples of this data set.

But, we do note a couple of patterns. The design-based IRS analysis
shows the largest standard error: a likely reason is that this is the
only approach that does not incorporate any spatial information
regarding mercury concentration across the contiguous United States. We
also see that both approaches using the GRTS sample have a lower
standard error than the both approaches using the IRS sample. We would
expect this to be the case for most samples because mercury
concentration exhibits spatial patterning, so a spatially balanced
sample should usually yield a lower standard error.

\hypertarget{sec:discussion}{%
\section{Discussion}\label{sec:discussion}}

The design-based and model-based approaches to inference are
fundamentally different paradigms by which to select samples and analyze
data. The design-based approach incorporates randomness through sampling
to estimate a population parameter. The model-based approach
incorporates randomness through distributional assumptions to predict
the realized value of a random variable. Though these approaches have
often been compared in the literature both from theoretical and
analytical perspectives, our contribution lies in studying them in a
spatial context while implementing spatially balanced sampling. Aside
from the theoretical differences described, a few analytical findings
were particularly notable: the design decision (GRTS vs IRS) seems much
more important than the analysis decision (design-based vs model-based);
Independent of the analysis approach, there is no reason to prefer IRS
over GRTS for spatial data -- GRTS tends to perform at least as well as
IRS when there is no spatial correlation increasingly than IRS as the
strength of spatial correlation increases; the gap in relative
efficiency between GRTS-design and GRTS-model widens as the strength of
spatial correlation increases; and when the data are skewed, interval
coverage for all approaches improves both as the sample size increases
and as the strength of correlation increases.

There are several benefits and drawbacks of the design-based and
model-based approaches for spatial data, some of which we have not yet
discussed but are worthy of consideration in future research. The
design-based approach relies on few assumptions, while the model-based
approach relies on rigid distributional ones. The Horvitz-Thompson
estimator of means and totals is unbiased and computationally efficient,
while model-based estimation of covariance parameters can be
computationally burdensome, especially for likelihood-based methods such
as REML that rely on inverting a covariance matrix. The design-based
approach also more naturally handles binary data, free from the more
complicated logistic regression formulation commonly used to handle
binary data in a model-based approach. The model-based approach,
however, can quantify the relationship between covariates (predictor
variables) and the response variable, something the design-based
approach cannot do naturally. The model-based approach also yields
estimated spatial covariance parameters, which help better understand
the process of study. Model selection is also possible using model-based
approaches and criteria such as likelihood ratio tests or AIC (Akaike,
1974). Model-based approaches are capable of more efficient small-area
estimation than design-based approaches by leveraging distributional
assumptions in areas with few observed sites. Model-based approaches can
also compute site-by-site predictions at unobserved locations and use
them to construct informative visualizations. The benefits and drawbacks
of both approaches, alongside our theoretical and analytical
comparisons, should be heavily considered when choosing among them. This
is especially true from an analysis perspective, as we found that using
a spatially balanced sampling algorithm benefits both design-based and
model-based analyses.

\hypertarget{references}{%
\section*{References}\label{references}}
\addcontentsline{toc}{section}{References}

\hypertarget{refs}{}
\begin{CSLReferences}{1}{0}
\leavevmode\hypertarget{ref-akaike1974new}{}%
Akaike, H., 1974. A new look at the statistical model identification.
IEEE transactions on automatic control 19, 716--723.

\leavevmode\hypertarget{ref-barabesi2011sampling}{}%
Barabesi, L., Franceschi, S., 2011. Sampling properties of spatial total
estimators under tessellation stratified designs. Environmetrics 22,
271--278.

\leavevmode\hypertarget{ref-benedetti2017spatially}{}%
Benedetti, R., Piersimoni, F., 2017. A spatially balanced design with
probability function proportional to the within sample distance.
Biometrical Journal 59, 1067--1084.

\leavevmode\hypertarget{ref-benedetti2017spatiallyreview}{}%
Benedetti, R., Piersimoni, F., Postiglione, P., 2017. Spatially balanced
sampling: A review and a reappraisal. International Statistical Review
85, 439--454.

\leavevmode\hypertarget{ref-breiman2001random}{}%
Breiman, L., 2001. Random forests. Machine learning 45, 5--32.

\leavevmode\hypertarget{ref-brus1997random}{}%
Brus, D., De Gruijter, J., 1997. Random sampling or geostatistical
modelling? Choosing between design-based and model-based sampling
strategies for soil (with discussion). Geoderma 80, 1--44.

\leavevmode\hypertarget{ref-brus2020statistical}{}%
Brus, D.J., 2020. Statistical approaches for spatial sample survey:
Persistent misconceptions and new developments. European Journal of Soil
Science.

\leavevmode\hypertarget{ref-chan2020bayesian}{}%
Chan-Golston, A.M., Banerjee, S., Handcock, M.S., 2020. Bayesian
inference for finite populations under spatial process settings.
Environmetrics 31, e2606.

\leavevmode\hypertarget{ref-chiles1999geostatistics}{}%
Chiles, J.-P., Delfiner, P., 1999. Geostatistics: {Modeling Spatial
Uncertainty}. {John Wiley \& Sons}, New York.

\leavevmode\hypertarget{ref-cicchitelli2012model}{}%
Cicchitelli, G., Montanari, G.E., 2012. Model-assisted estimation of a
spatial population mean. International Statistical Review 80, 111--126.

\leavevmode\hypertarget{ref-cooper2006sampling}{}%
Cooper, C., 2006. Sampling and variance estimation on continuous
domains. Environmetrics: The official journal of the International
Environmetrics Society 17, 539--553.

\leavevmode\hypertarget{ref-cressie1993statistics}{}%
Cressie, N., 1993. Statistics for spatial data. John Wiley \& Sons.

\leavevmode\hypertarget{ref-de1990model}{}%
De Gruijter, J., Ter Braak, C., 1990. Model-free estimation from spatial
samples: A reappraisal of classical sampling theory. Mathematical
geology 22, 407--415.

\leavevmode\hypertarget{ref-diggle2010geostatistical}{}%
Diggle, P.J., Menezes, R., Su, T., 2010. Geostatistical inference under
preferential sampling. Journal of the Royal Statistical Society: Series
C (Applied Statistics) 59, 191--232.

\leavevmode\hypertarget{ref-dumelle2021spsurvey}{}%
Dumelle, M., Olsen, A.R., Kincaid, T., Weber, M., 2021. Selecting and
analyzing spatial probability samples in r using spsurvey. Manuscript
Submitted for Publication.

\leavevmode\hypertarget{ref-fix1951discriminatory}{}%
Fix, E., Hodges, J.L., 1951. Discriminatory analysis, nonparametric
discrimination: Consistency properties. USAF School of Aviation
Medicine.

\leavevmode\hypertarget{ref-grafstrom2012spatiallypoisson}{}%
Grafström, A., 2012. Spatially correlated poisson sampling. Journal of
Statistical Planning and Inference 142, 139--147.

\leavevmode\hypertarget{ref-grafstrom2013well}{}%
Grafström, A., Lundström, N.L., 2013. Why well spread probability
samples are balanced. Open Journal of Statistics 3, 36--41.

\leavevmode\hypertarget{ref-grafstrom2012spatially}{}%
Grafström, A., Lundström, N.L., Schelin, L., 2012. Spatially balanced
sampling through the pivotal method. Biometrics 68, 514--520.

\leavevmode\hypertarget{ref-grafstrom2018spatially}{}%
Grafström, A., Matei, A., 2018. Spatially balanced sampling of
continuous populations. Scandinavian Journal of Statistics 45, 792--805.

\leavevmode\hypertarget{ref-hansen1983evaluation}{}%
Hansen, M.H., Madow, W.G., Tepping, B.J., 1983. An evaluation of
model-dependent and probability-sampling inferences in sample surveys.
Journal of the American Statistical Association 78, 776--793.

\leavevmode\hypertarget{ref-higham2021sptotal}{}%
Higham, M., Ver Hoef, J., Frank, B., Dumelle, M., 2021. Sptotal:
Predicting totals and weighted sums from spatial data.

\leavevmode\hypertarget{ref-horvitz1952generalization}{}%
Horvitz, D.G., Thompson, D.J., 1952. A generalization of sampling
without replacement from a finite universe. Journal of the American
statistical Association 47, 663--685.

\leavevmode\hypertarget{ref-lohr2009sampling}{}%
Lohr, S.L., 2009. Sampling: Design and analysis. Nelson Education.

\leavevmode\hypertarget{ref-robertson2013bas}{}%
Robertson, B., Brown, J., McDonald, T., Jaksons, P., 2013. BAS: Balanced
acceptance sampling of natural resources. Biometrics 69, 776--784.

\leavevmode\hypertarget{ref-robertson2018halton}{}%
Robertson, B., McDonald, T., Price, C., Brown, J., 2018. Halton
iterative partitioning: Spatially balanced sampling via partitioning.
Environmental and Ecological Statistics 25, 305--323.

\leavevmode\hypertarget{ref-sarndal2003model}{}%
Särndal, C.-E., Swensson, B., Wretman, J., 2003. Model assisted survey
sampling. Springer Science \& Business Media.

\leavevmode\hypertarget{ref-schabenberger2017statistical}{}%
Schabenberger, O., Gotway, C.A., 2017. Statistical methods for spatial
data analysis. CRC press.

\leavevmode\hypertarget{ref-sen1953estimate}{}%
Sen, A.R., 1953. On the estimate of the variance in sampling with
varying probabilities. Journal of the Indian Society of Agricultural
Statistics 5, 127.

\leavevmode\hypertarget{ref-sterba2009alternative}{}%
Sterba, S.K., 2009. Alternative model-based and design-based frameworks
for inference from samples to populations: From polarization to
integration. Multivariate behavioral research 44, 711--740.

\leavevmode\hypertarget{ref-stevens2003variance}{}%
Stevens, D.L., Olsen, A.R., 2003. Variance estimation for spatially
balanced samples of environmental resources. Environmetrics 14,
593--610.

\leavevmode\hypertarget{ref-stevens2004spatially}{}%
Stevens, D.L., Olsen, A.R., 2004. Spatially balanced sampling of natural
resources. Journal of the american Statistical association 99, 262--278.

\leavevmode\hypertarget{ref-verhoef2002sampling}{}%
Ver Hoef, J., 2002. Sampling and geostatistics for spatial data.
Ecoscience 9, 152--161.

\leavevmode\hypertarget{ref-verhoef2008spatial}{}%
Ver Hoef, J.M., 2008. Spatial methods for plot-based sampling of
wildlife populations. Environmental and Ecological Statistics 15, 3--13.

\leavevmode\hypertarget{ref-ver2013comparison}{}%
Ver Hoef, J.M., Temesgen, H., 2013. A comparison of the spatial linear
model to nearest neighbor (k-NN) methods for forestry applications. PloS
one 8, e59129.

\leavevmode\hypertarget{ref-wang2013design}{}%
Wang, J.-F., Jiang, C.-S., Hu, M.-G., Cao, Z.-D., Guo, Y.-S., Li, L.-F.,
Liu, T.-J., Meng, B., 2013. Design-based spatial sampling: Theory and
implementation. Environmental modelling \& software 40, 280--288.

\leavevmode\hypertarget{ref-wang2012review}{}%
Wang, J.-F., Stein, A., Gao, B.-B., Ge, Y., 2012. A review of spatial
sampling. Spatial Statistics 2, 1--14.

\end{CSLReferences}


\end{document}
