\documentclass[]{elsarticle} %review=doublespace preprint=single 5p=2 column
%%% Begin My package additions %%%%%%%%%%%%%%%%%%%
\usepackage[hyphens]{url}

  \journal{An awesome journal} % Sets Journal name


\usepackage{lineno} % add
\providecommand{\tightlist}{%
  \setlength{\itemsep}{0pt}\setlength{\parskip}{0pt}}

\usepackage{graphicx}
\usepackage{booktabs} % book-quality tables
%%%%%%%%%%%%%%%% end my additions to header

\usepackage[T1]{fontenc}
\usepackage{lmodern}
\usepackage{amssymb,amsmath}
\usepackage{ifxetex,ifluatex}
\usepackage{fixltx2e} % provides \textsubscript
% use upquote if available, for straight quotes in verbatim environments
\IfFileExists{upquote.sty}{\usepackage{upquote}}{}
\ifnum 0\ifxetex 1\fi\ifluatex 1\fi=0 % if pdftex
  \usepackage[utf8]{inputenc}
\else % if luatex or xelatex
  \usepackage{fontspec}
  \ifxetex
    \usepackage{xltxtra,xunicode}
  \fi
  \defaultfontfeatures{Mapping=tex-text,Scale=MatchLowercase}
  \newcommand{\euro}{€}
\fi
% use microtype if available
\IfFileExists{microtype.sty}{\usepackage{microtype}}{}
\bibliographystyle{elsarticle-harv}
\ifxetex
  \usepackage[setpagesize=false, % page size defined by xetex
              unicode=false, % unicode breaks when used with xetex
              xetex]{hyperref}
\else
  \usepackage[unicode=true]{hyperref}
\fi
\hypersetup{breaklinks=true,
            bookmarks=true,
            pdfauthor={},
            pdftitle={A comparison of model-based and design-based approaches for spatial data.},
            colorlinks=false,
            urlcolor=blue,
            linkcolor=magenta,
            pdfborder={0 0 0}}
\urlstyle{same}  % don't use monospace font for urls

\setcounter{secnumdepth}{5}
% Pandoc toggle for numbering sections (defaults to be off)

% Pandoc citation processing

% Pandoc header

\usepackage{bm} \usepackage{bbm} \usepackage{color} \DeclareMathOperator{\var}{{var}} \DeclareMathOperator{\cov}{{cov}}

\begin{document}
\begin{frontmatter}

  \title{A comparison of model-based and design-based approaches for spatial
data.}
    \author[USEPA]{Michael Dumelle\corref{1}}
   \ead{Dumelle.Michael@epa.gov} 
    \author[STLAW]{Matthew Higham\corref{1}}
   \ead{mhigham@stlaw.edu} 
    \author[OSU]{Lisa Madsen}
  
    \author[USEPA]{Anthony R. Olsen}
  
    \author[NOAA]{Jay M. Ver Hoef}
  
      \address[USEPA]{United States Environmental Protection Agency, 200 SW 35th St,
Corvallis, Oregon, 97333}
    \address[STLAW]{Saint Lawrence University Department of Math, Computer Science, and
Statistics, 23 Romoda Drive, Canton, New York, 13617}
    \address[OSU]{Oregon State University Department of Statistics, 239 Weniger Hall,
Corvallis, Oregon, 97331}
    \address[NOAA]{Marine Mammal Laboratory, Alaska Fisheries Science Center, National
Oceanic and Atmospheric Administration, Seattle, Washington, 98115}
      \cortext[1]{Corresponding Author}
  
  \begin{abstract}
  This is the abstract.
  
  It consists of two paragraphs.
  \end{abstract}
  
 \end{frontmatter}

\emph{Text based on elsarticle sample manuscript, see
\url{http://www.elsevier.com/author-schemas/latex-instructions\#elsarticle}}

\hypertarget{introduction}{%
\section{Introduction}\label{introduction}}

Please leave comments in your
\textcolor{blue}{c}\textcolor{magenta}{o}\textcolor{green}{l}\textcolor{red}{o}\textcolor{cyan}{r}:
\textcolor{blue}{Michael}, \textcolor{magenta}{Matt},
\textcolor{green}{Lisa}, \textcolor{red}{Tony}, \textcolor{cyan}{Jay}.

\hypertarget{background}{%
\section{Background}\label{background}}

\hypertarget{design-based-philosphy}{%
\subsection{Design-Based Philosphy}\label{design-based-philosphy}}

Design-Based Overview

Design-based inference uses characteristics of the sampling design to to
estimate parameters of interest Typically, there are few assumptions
involved because intervals are derived using the sampling design
itself\ldots\ldots\ldots..

\hypertarget{model-based-philosphy}{%
\subsection{Model-Based Philosphy}\label{model-based-philosphy}}

Model-Based Overview

On the other hand, model-based inference imposes additional assumptions
on the data with a potential to provide more precise estimates if the
additional assumptions hold. Instead of estimating true but unknown
parameters, the goal of model-based inference in the spatial context is
often \emph{prediction} of an unknown quantity. This is a fundamental
philosophical difference between sampling-based and model-based
approaches. Instead of \emph{estimating} a fixed unknown mean, we are
\emph{predicting} the value of the mean, a random variable. We know that
if we sampled all sites, we would have an exact prediction for the mean
of our one realized spatial surface, without any uncertainty. But, the
true mean of the spatial process that generated our realized data is
still not known, and, in the prediction context, we typically do not
care much about what value the mean of the underlying process takes.

Figure 1a. Data is fixed. In a finite population example, show a 3d
surface that can be generated by anything. If we repeatedly sample the
surface, then 95\% of all 95\% CIs will contain the true mean, which
never changes.

Figure 1b. Spatial process is fixed. In a finite population example,
show 10 3d surfaces that are generated from some model. If we repeatedly
generate the surface and obtain a sample, then 95\% of all 95\% PIs will
contain the realized means. The realized mean changes from surface to
surface and it's not necessarily the case that 95\% of all 95\% PIs will
contain the true, underlying mean.

\hypertarget{comparing-design-based-vs.-model-based}{%
\subsection{Comparing Design-Based
vs.~Model-Based}\label{comparing-design-based-vs.-model-based}}

Design-Based and Model-Based Comparisons

There have been many comparisons between the two paradigms\ldots\ldots.

\hypertarget{spatially-balanced-design-and-analysis}{%
\subsection{Spatially Balanced Design and
Analysis}\label{spatially-balanced-design-and-analysis}}

Spatially balanced sampling algorithms use spatial information to obtain
samples spread out in space. Spatially balanced samples are useful
because they tend to yield estimators that are more precise than
estimators constructed from an sampling algorithm that is not spatially
balanced ((Barabesi and Franceschi, 2011; Benedetti et al., 2017;
Grafström and Lundström, 2013; Robertson et al., 2013; Stevens Jr and
Olsen, 2004; Wang et al., 2013)). Many spatially balanced sampling
algorithms exist, including the Generalized Random Tessellation
Stratified (Stevens Jr and Olsen, 2004), the Local Pivotal Method
(Grafström et al., 2012; Grafström and Matei, 2018), Spatially
Correlated Poisson Sampling (Grafström, 2012), Balanced Acceptance
Sampling (Robertson et al., 2013), Within-Sample-Distance (Benedetti and
Piersimoni, 2017), and Halton Iterative Partitioning (Robertson et al.,
2018) algorithms. Here we focus on the Generalized Random Tessellation
Stratified (GRTS) algorithm, which has several attractive properties
that we discuss next.

The GRTS algorithm is used to sample from finite and infinite sample
frames. A finite sample frames contains a finite number of sampling
units and is related to a point geometry. An infinite sample frame
contains an infinite number of sampling units and is related to linear
and polygon geometries. Examples of point, linear, and polygon resources
include lake centroids, stream networks, and wetland areas,
respectively. In addition to its applicability for finite and infinite
sample frames, the GRTS algorithm naturally accommodates stratified
designs and designs with unequal selection probabilities. The algorithm
has also been used to select replacement sites using reverse
hierarchical ordering (Stevens Jr and Olsen, 2004). Replacement sites
are used to replace sites in the original sample that cannot be sampled,
often as a result of physical difficulty in reaching the site or
landowner denial of access to the sites. More recently, the GRTS
algorithm also accommodates legacy (historical) sites, minimum distance
between sites, and nearest neighbor replacement sites. The GRTS
algorithm is implemented in the \textbf{\textsf{R}} package
\texttt{spsurvey} (Dumelle et al., 2021).

accomodates stratification, unequal selection probabilities, oversample
sites It has also recently been updated to accommodate legacy
(historical) sites, minimum distance between sites, and a nearest
neighbor replacement sites comp efficient software

\hypertarget{finite-population-block-kriging}{%
\subsection{Finite Population Block
Kriging}\label{finite-population-block-kriging}}

Finite Population Block Kriging (FPBK) is an alternative to
samipling-based methods (Ver Hoef, 2008). FPBK expands the
geostatistical kriging framework to the finite population setting.
Instead of basing inference off of a specific sampling design, we assume
the data were generated by a spatial process with parameters that can be
estimated using the framework of a model.

Ver Hoef (2008) gives details on the theory of FPBK, but some of the
basic principles are summarized below. For a response variable
\(\mathbf{z}\) that can be measured on a finite number of \(N\) sites,
we want to predict some linear function of the response variable,
\(\tau(\mathbf{z}) = \mathbf{b}^\prime \mathbf{z}\), where
\(\mathbf{b}\) is a vector of weights. For example, if we want to
predict the total abundance across all sites, then we would use a vector
of 1's for the weights.

Typically, however, we only have a sample of the \(N\) sites. Denoting
quantities that are part of the sampled sites with a subscript \emph{s}
and quantities that are part of the unsampled sites with a subscript
\emph{u},

\begin{equation}
\begin{pmatrix} \label{equation:Zmarginal}
    \mathbf{z}_s      \\
    \mathbf{z}_u
\end{pmatrix}
=
\begin{pmatrix}
  \mathbf{X}_s    \\
  \mathbf{X}_u
\end{pmatrix}
\bm{\beta} +
\begin{pmatrix}
\bm{\delta}_s    \\
\bm{\delta}_u
\end{pmatrix},
\end{equation} where \(\mathbf{X}_s\) and \(\mathbf{X}_u\) are the
design matrices for the sampled and unsampled sites, respectively, and
\(\bm{\delta}_s\) and \(\bm{\delta}_u\) are random errors for the
sampled and unsampled sites. Denoting
\(\bm{\delta} \equiv [\bm{\delta}_s \,\, \bm{\delta}_u]'\), we assume
that \(E(\bm{\delta})\) = \(\mathbf{0}\).

We also typically assume that there is spatial correlation in
\(\bm{\delta}\), which can be modeled using a covariance function. Many
common choices for this function assume that spatial covariance
decreases with increasing Euclidean distance between sites. The primary
function used throughout the simulations and applications of this
manuscript is the Exponential covariance function: the \(i,j^{th}\)
entry for \(\mathop{\mathrm{{var}}}(\bm{\delta})\) is \mbox{}
\begin{align}\label{equation:expcov}
\mathop{\mathrm{{cov}}}(\delta_i, \delta_j) = \theta_3 + \theta_1\exp(-h_{i,j}/\theta_2), \\
\textcolor{blue}{\sigma^2[(1 - v)\exp(-3\mathbf{h}_{i,j}/\phi) + v\mathbbm{1}\{\mathbf{h}_{i,j} = 0\}]}, 
\end{align} where \(h_{i,j}\) is the distance between sites \(i\) and
\(j\), and \(\bm{\theta}\) is a vector of spatial covariance parameters
of the partial sill \(\theta_1\), the range \(\theta_2\), and the nugget
\(\theta_3\). However, any spatial covariance function could be used in
the place of the Exponential, including functions that allow for
anisotropy {[}pg. 80 - 93{]}(Chiles and Delfiner, 1999).

With the above model formulation, the Best Linear Unbiased Predictor
(BLUP) for \(\tau(\mathbf{b}'\mathbf{z})\)
\textcolor{blue}{Did you mean to give the form of the BLUP here? $\tau(\mathbf{b}'\mathbf{z})$ is vague}
and its prediction variance can be computed. While details of the
derivation are in (Ver Hoef, 2008), we note here that the predictor and
its variance are both moment-based. Neither require a particular
distribution for \(\mathbf{z}\).

\hypertarget{numerical-study}{%
\section{Numerical Study}\label{numerical-study}}

\hypertarget{software}{%
\subsection{Software}\label{software}}

FPBK can be readily performed in \texttt{R} with the \texttt{sptotal}
package (Higham et al., 2020). We use \texttt{sptotal} for both the
simulation analysis and the application, estimating parameters with
Restricted Maximum Likelihood (REML).

\hypertarget{discussion}{%
\section{Discussion}\label{discussion}}

\hypertarget{references}{%
\section*{References}\label{references}}
\addcontentsline{toc}{section}{References}

\hypertarget{refs}{}
\leavevmode\hypertarget{ref-barabesi2011sampling}{}%
Barabesi, L., Franceschi, S., 2011. Sampling properties of spatial total
estimators under tessellation stratified designs. Environmetrics 22,
271--278.

\leavevmode\hypertarget{ref-benedetti2017spatially}{}%
Benedetti, R., Piersimoni, F., 2017. A spatially balanced design with
probability function proportional to the within sample distance.
Biometrical Journal 59, 1067--1084.

\leavevmode\hypertarget{ref-benedetti2017spatiallyreview}{}%
Benedetti, R., Piersimoni, F., Postiglione, P., 2017. Spatially balanced
sampling: A review and a reappraisal. International Statistical Review
85, 439--454.

\leavevmode\hypertarget{ref-chiles1999geostatistics}{}%
Chiles, J.-P., Delfiner, P., 1999. Geostatistics: Modeling Spatial
Uncertainty. John Wiley \& Sons, New York.

\leavevmode\hypertarget{ref-dumelle2021spsurvey}{}%
Dumelle, M., Olsen, A.R., Kincaid, T., Weber, M., 2021. Selecting and
analyzing spatial probability samples in r using spsurvey. Manuscript
Submitted for Publication.

\leavevmode\hypertarget{ref-grafstrom2012spatiallypoisson}{}%
Grafström, A., 2012. Spatially correlated poisson sampling. Journal of
Statistical Planning and Inference 142, 139--147.

\leavevmode\hypertarget{ref-grafstrom2013well}{}%
Grafström, A., Lundström, N.L., 2013. Why well spread probability
samples are balanced. Open Journal of Statistics 3, 36--41.

\leavevmode\hypertarget{ref-grafstrom2012spatially}{}%
Grafström, A., Lundström, N.L., Schelin, L., 2012. Spatially balanced
sampling through the pivotal method. Biometrics 68, 514--520.

\leavevmode\hypertarget{ref-grafstrom2018spatially}{}%
Grafström, A., Matei, A., 2018. Spatially balanced sampling of
continuous populations. Scandinavian Journal of Statistics 45, 792--805.

\leavevmode\hypertarget{ref-higham2020sptotal}{}%
Higham, M., Ver Hoef, J., Bryce, F., 2020. Sptotal: Predicting totals
and weighted sums from spatial data.

\leavevmode\hypertarget{ref-robertson2013bas}{}%
Robertson, B., Brown, J., McDonald, T., Jaksons, P., 2013. BAS: Balanced
acceptance sampling of natural resources. Biometrics 69, 776--784.

\leavevmode\hypertarget{ref-robertson2018halton}{}%
Robertson, B., McDonald, T., Price, C., Brown, J., 2018. Halton
iterative partitioning: Spatially balanced sampling via partitioning.
Environmental and Ecological Statistics 25, 305--323.

\leavevmode\hypertarget{ref-stevens2004spatially}{}%
Stevens Jr, D.L., Olsen, A.R., 2004. Spatially balanced sampling of
natural resources. Journal of the american Statistical association 99,
262--278.

\leavevmode\hypertarget{ref-verhoef2008spatial}{}%
Ver Hoef, J.M., 2008. Spatial methods for plot-based sampling of
wildlife populations. Environmental and Ecological Statistics 15, 3--13.

\leavevmode\hypertarget{ref-wang2013design}{}%
Wang, J.-F., Jiang, C.-S., Hu, M.-G., Cao, Z.-D., Guo, Y.-S., Li, L.-F.,
Liu, T.-J., Meng, B., 2013. Design-based spatial sampling: Theory and
implementation. Environmental modelling \& software 40, 280--288.


\end{document}


