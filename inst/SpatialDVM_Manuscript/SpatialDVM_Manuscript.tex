\documentclass[]{elsarticle} %review=doublespace preprint=single 5p=2 column
%%% Begin My package additions %%%%%%%%%%%%%%%%%%%
\usepackage[hyphens]{url}

  \journal{An awesome journal} % Sets Journal name


\usepackage{lineno} % add
\providecommand{\tightlist}{%
  \setlength{\itemsep}{0pt}\setlength{\parskip}{0pt}}

\usepackage{graphicx}
\usepackage{booktabs} % book-quality tables
%%%%%%%%%%%%%%%% end my additions to header

\usepackage[T1]{fontenc}
\usepackage{lmodern}
\usepackage{amssymb,amsmath}
\usepackage{ifxetex,ifluatex}
\usepackage{fixltx2e} % provides \textsubscript
% use upquote if available, for straight quotes in verbatim environments
\IfFileExists{upquote.sty}{\usepackage{upquote}}{}
\ifnum 0\ifxetex 1\fi\ifluatex 1\fi=0 % if pdftex
  \usepackage[utf8]{inputenc}
\else % if luatex or xelatex
  \usepackage{fontspec}
  \ifxetex
    \usepackage{xltxtra,xunicode}
  \fi
  \defaultfontfeatures{Mapping=tex-text,Scale=MatchLowercase}
  \newcommand{\euro}{€}
\fi
% use microtype if available
\IfFileExists{microtype.sty}{\usepackage{microtype}}{}
\bibliographystyle{elsarticle-harv}
\ifxetex
  \usepackage[setpagesize=false, % page size defined by xetex
              unicode=false, % unicode breaks when used with xetex
              xetex]{hyperref}
\else
  \usepackage[unicode=true]{hyperref}
\fi
\hypersetup{breaklinks=true,
            bookmarks=true,
            pdfauthor={},
            pdftitle={Short Paper},
            colorlinks=false,
            urlcolor=blue,
            linkcolor=magenta,
            pdfborder={0 0 0}}
\urlstyle{same}  % don't use monospace font for urls

\setcounter{secnumdepth}{5}
% Pandoc toggle for numbering sections (defaults to be off)

% Pandoc citation processing
\newlength{\csllabelwidth}
\setlength{\csllabelwidth}{3em}
\newlength{\cslhangindent}
\setlength{\cslhangindent}{1.5em}
% for Pandoc 2.8 to 2.10.1
\newenvironment{cslreferences}%
  {}%
  {\par}
% For Pandoc 2.11+
\newenvironment{CSLReferences}[3] % #1 hanging-ident, #2 entry spacing
 {% don't indent paragraphs
  \setlength{\parindent}{0pt}
  % turn on hanging indent if param 1 is 1
  \ifodd #1 \everypar{\setlength{\hangindent}{\cslhangindent}}\ignorespaces\fi
  % set entry spacing
  \ifnum #2 > 0
  \setlength{\parskip}{#2\baselineskip}
  \fi
 }%
 {}
\usepackage{calc} % for calculating minipage widths
\newcommand{\CSLBlock}[1]{#1\hfill\break}
\newcommand{\CSLLeftMargin}[1]{\parbox[t]{\csllabelwidth}{#1}}
\newcommand{\CSLRightInline}[1]{\parbox[t]{\linewidth - \csllabelwidth}{#1}}
\newcommand{\CSLIndent}[1]{\hspace{\cslhangindent}#1}

% Pandoc header

\usepackage{bm} \DeclareMathOperator{\var}{{var}} \DeclareMathOperator{\cov}{{cov}}

\begin{document}
\begin{frontmatter}

  \title{Short Paper}
    \author[Some Institute of Technology]{Alice Anonymous\corref{1}}
   \ead{alice@example.com} 
    \author[Another University]{Bob Security}
   \ead{bob@example.com} 
    \author[Another University]{Cat Memes\corref{2}}
   \ead{cat@example.com} 
    \author[Some Institute of Technology]{Derek Zoolander\corref{2}}
   \ead{derek@example.com} 
      \address[Some Institute of Technology]{Department, Street, City,
State, Zip}
    \address[Another University]{Department, Street, City, State, Zip}
      \cortext[1]{Corresponding Author}
    \cortext[2]{Equal contribution}
  
  \begin{abstract}
  This is the abstract.

  It consists of two paragraphs.
  \end{abstract}
  
 \end{frontmatter}

\emph{Text based on elsarticle sample manuscript, see
\url{http://www.elsevier.com/author-schemas/latex-instructions\#elsarticle}}

\hypertarget{introduction}{%
\section{Introduction}\label{introduction}}

\hypertarget{design-based-philosphy}{%
\subsection{Design-Based Philosphy}\label{design-based-philosphy}}

\hypertarget{model-based-philosphy}{%
\subsection{Model-Based Philosphy}\label{model-based-philosphy}}

On the other hand, model-based inference imposes additional assumptions
on the data with a potential to provide more precise estimates if the
additional assumptions hold. Instead of estimating true but unknown
parameters, the goal of model-based inference in the spatial context is
often \emph{prediction} of an unknown quantity. This is a fundamental
philosophical difference between sampling-based and model-based
approaches. Instead of \emph{estimating} an fixed unknown mean, we are
\emph{predicting} the value of the mean, a random variable. We know that
if we sampled all sites, we would have an exact prediction for the mean
of our one realized spatial surface, without any uncertainty. But, the
true mean of the spatial process that generated our realized data is
still not known, and, in the prediction context, we typically do not
care much about what value the mean of the underlying process takes.

Figure 1a. Data is fixed. In a finite population example, show a 3d
surface that can be generated by anything. If we repeatedly sample the
surface, then 95\% of all 95\% CIs will contain the true mean, which
never changes.

Figure 1b. Spatial process is fixed. In a finite population example,
show 10 3d surfaces that are generated from some model. If we repeatedly
generate the surface and obtain a sample, then 95\% of all 95\% PIs will
contain the realized means. The realized mean changes from surface to
surface and it's not necessarily the case that 95\% of all 95\% PIs will
contain the true, underlying mean.

\hypertarget{comparing-design-based-vs.-model-based}{%
\subsection{Comparing Design-Based
vs.~Model-Based}\label{comparing-design-based-vs.-model-based}}

There have been many comparisons between the two paradigms\ldots\ldots.

\hypertarget{spatially-balanced-design-and-analysis}{%
\subsection{Spatially Balanced Design and
Analysis}\label{spatially-balanced-design-and-analysis}}

\hypertarget{finite-population-block-kriging}{%
\subsection{Finite Population Block
Kriging}\label{finite-population-block-kriging}}

Finite Population Block Kriging (FPBK) is an alternative to
samipling-based methods (Ver Hoef, 2008). FPBK expands the
geostatistical kriging framework to the finite population setting.
Instead of relying on a specific sampling design, we assume the data
were produced by a spatial stochastic process with spatial parameters
that can be estimated.

Ver Hoef (2008) gives details on the theory of FPBK, but some of the
basic principles are summarized below. For a response variable
\(\mathbf{z}\) that can be measured on a finite number of \(N\) sites,
our goal is to predict some linear function of all of the sample units,
\(\tau(\mathbf{z}) = \mathbf{b}^\prime \mathbf{z}\), where
\(\mathbf{b}\) is a vector of weights. One common vector of weights is a
vector of 1's to predict the total abundance in the region.

Typically, however, we only have a sample of the \(N\) sites. Denoting
quantities that are part of the sampled sites with a subscript \emph{s}
and quantities that are part of the unsampled sites with a subscript
\emph{u},

\begin{equation}
\begin{pmatrix} \label{equation:Zmarginal}
    \mathbf{z}_s      \\
    \mathbf{z}_u
\end{pmatrix}
=
\begin{pmatrix}
  \mathbf{X}_s    \\
  \mathbf{X}_u
\end{pmatrix}
\bm{\beta} +
\begin{pmatrix}
\bm{\delta}_s    \\
\bm{\delta}_u
\end{pmatrix},
\end{equation} where \(\mathbf{X}_s\) and \(\mathbf{X}_u\) are the
design matrices for the sampled and unsampled sites, respectively, and
\(\bm{\delta}_s\) and \(\bm{\delta}_u\) are zero-mean random errors for
the sampled and unsampled sites. Denoting
\(\bm{\delta} \equiv [\bm{\delta}_s \,\, \bm{\delta}_u]'\), then we
assume that \(E(\bm{\delta})\) = \(\mathbf{0}\).

We also typically assume that there is spatial correlation in
\(\bm{\delta}\), which can be modeled using a covariance function. Many
common choices for this function assume that spatial covariance
decreases with increasing Euclidean distance between sites. The primary
function used throughout the simulations and applications of this
manuscript is the Exponential covariance function: the \(i,j^{th}\)
entry for \(\mathop{\mathrm{{var}}}(\bm{\delta})\) is \mbox{}
\begin{equation}
\mathop{\mathrm{{cov}}}(\delta_i, \delta_j) = \theta_3 + \theta_1\exp(-h_{i,j}/\theta_2), \label{equation:expcov}
\end{equation} where \(h_{i,j}\) is the distance between sites \(i\) and
\(j\), and \(\bm{\theta}\) is a vector of spatial covariance parameters
of the partial sill \(\theta_1\), the range \(\theta_2\), and the nugget
\(\theta_3\). However, any spatial covariance function could be used in
the place of the Exponential, including functions that allow for
anisotropy {[}pg. 80 - 93{]}(Chiles and Delfiner, 1999).

With the above model formulation, the Best Linear Unbiased Predictor
(BLUP) for \(\tau(\mathbf{b}'\mathbf{z})\) and its prediction variance
can be computed. While details of the derivation are in (Ver Hoef,
2008), we note here that the predictor and its variance are both
moment-based. Neither require a particular distribution for
\(\mathbf{z}\).

\textbf{Software Implementation}: This probably goes in a different spot
(right before simulations?), but putting it here while it's on my mind.

FPBK can be readily performed in \texttt{R} with the \texttt{sptotal}
package (Matt et al., 2020). We use \texttt{sptotal} for both the
simulation analysis and the application, estimating parameters with
Restricted Maximum Likelihood (REML).

\hypertarget{installation}{%
\paragraph{Installation}\label{installation}}

If the document class \emph{elsarticle} is not available on your
computer, you can download and install the system package
\emph{texlive-publishers} (Linux) or install the LaTeX package
\emph{elsarticle} using the package manager of your TeX installation,
which is typically TeX Live or MikTeX.

\hypertarget{usage}{%
\paragraph{Usage}\label{usage}}

Once the package is properly installed, you can use the document class
\emph{elsarticle} to create a manuscript. Please make sure that your
manuscript follows the guidelines in the Guide for Authors of the
relevant journal. It is not necessary to typeset your manuscript in
exactly the same way as an article, unless you are submitting to a
camera-ready copy (CRC) journal.

\hypertarget{functionality}{%
\paragraph{Functionality}\label{functionality}}

The Elsevier article class is based on the standard article class and
supports almost all of the functionality of that class. In addition, it
features commands and options to format the

\begin{itemize}
\item
  document style
\item
  baselineskip
\item
  front matter
\item
  keywords and MSC codes
\item
  theorems, definitions and proofs
\item
  lables of enumerations
\item
  citation style and labeling.
\end{itemize}

\hypertarget{front-matter}{%
\section{Front matter}\label{front-matter}}

The author names and affiliations could be formatted in two ways:

\begin{enumerate}
\def\labelenumi{(\arabic{enumi})}
\item
  Group the authors per affiliation.
\item
  Use footnotes to indicate the affiliations.
\end{enumerate}

See the front matter of this document for examples. You are recommended
to conform your choice to the journal you are submitting to.

\hypertarget{bibliography-styles}{%
\section{Bibliography styles}\label{bibliography-styles}}

There are various bibliography styles available. You can select the
style of your choice in the preamble of this document. These styles are
Elsevier styles based on standard styles like Harvard and Vancouver.
Please use BibTeX~to generate your bibliography and include DOIs
whenever available.

Here are two sample references: (\textbf{Feynman1963118?};
\textbf{Dirac1953888?}).

\hypertarget{references}{%
\section*{References}\label{references}}
\addcontentsline{toc}{section}{References}

\hypertarget{refs}{}
\begin{CSLReferences}{1}{0}
\leavevmode\hypertarget{ref-ChilesEtAl1999GeostatisticsModelingSpatial}{}%
Chiles, J.-P., Delfiner, P., 1999. Geostatistics: {Modeling Spatial
Uncertainty}. {John Wiley \& Sons}, New York.

\leavevmode\hypertarget{ref-sptotal2020}{}%
Matt, H., Jay, V.H., Bryce, F., 2020. Sptotal: Predicting totals and
weighted sums from spatial data.

\leavevmode\hypertarget{ref-VerHoef2008spatial}{}%
Ver Hoef, J.M., 2008. Spatial methods for plot-based sampling of
wildlife populations. Environmental and Ecological Statistics 15, 3--13.

\end{CSLReferences}


\end{document}

