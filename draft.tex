%%%%%%%%%%%%%%%%%%%%%%%%%%%%%%%%%
% Document Start
%%%%%%%%%%%%%%%%%%%%%%%%%%%%%%%%%

% this is where we read in all of the preamble material (packages, etc. - stuff that comes before the document begins)
% preamble

% setting the document font and class
\documentclass[12pt]{article}

%ams math packages
\usepackage{amsmath, amssymb, amsfonts, amsthm}

%jay ver hoef's macros
\usepackage{jvhmacros}

% uncomment the line below if you want 2 cm margins instead of wide margins - wide margins are 1in
%2 cm are smaller than 1in, so see what works after final version ready
\usepackage[margin=1in,heightrounded=true,centering]{geometry}


% forcing figure placement
\usepackage{float}

% putting in code from computing languages
\usepackage{listings}

% adding certain math scripts
\usepackage{mathrsfs}

% adding in R code
\usepackage{verbatim}

% adding certain math scripts
\usepackage{euscript}

% setting space between lines
\usepackage{setspace}

% provides striking out text
\usepackage{soul}

% adding line numbers
\usepackage{lineno}

% arguments to \includegraphics
\usepackage{graphicx}

% customize captions in floating environments (figure and table)
\usepackage{caption}

% envrionment for subfigures
\usepackage{subcaption}

% customization over enumerate, itemize, description
\usepackage[shortlabels]{enumitem}

% typesetting of \maketitle
\usepackage{titling}
% redefining \thanks
\makeatletter
\def\thanks#1{\protected@xdef\@thanks{\@thanks
        \protect\footnotetext{#1}}}
\makeatother

% implements citations \cite
\usepackage{natbib}
\setlength{\bibsep}{0.0pt}

% additional equation typesetting
\usepackage{mathtools}
% only show references used in equation numbers
\mathtoolsset{showonlyrefs=true}

% tables spanning multiple rows
\usepackage{multirow}

% make certain math fonts bold
\usepackage{bm, bbm}

% adding color customization
\usepackage[usenames, dvipsnames]{color}

% wrapping figures around text
\usepackage{wrapfig}

% table quality of life enhancements
\usepackage{booktabs}
% making table entries bold without making them larger
\newcommand{\bftab}{\fontseries{b}\selectfont}

% indents first paragraph after section header
\usepackage{indentfirst}

% allows for tables across successive pages
\usepackage{longtable}

% adding keywords
\providecommand{\keywords}[1]{\textbf{\text{Keywords---}} #1}

%JASA requirements for section numbering
\renewcommand\thesection{\arabic{section}.} \renewcommand\thesubsection{\thesection\arabic{subsection}}

% Inserts \clearpage before \begin{appendices}
\usepackage{etoolbox}
\usepackage{appendix}
\BeforeBeginEnvironment{appendices}{\clearpage}

% Colored References
\usepackage[colorlinks = true,
            linkcolor = blue,
            urlcolor  = blue,
            citecolor = blue,
            anchorcolor = blue]{hyperref}

% double spaced
\doublespacing
% influences line breaking
\tolerance=2000 
% used if TeX can't set the paragraph below the \tolerance badness
\emergencystretch=20pt


% adding author information
\title{\large \textbf{A Comparison of Design-Based and Model-Based Approaches for Spatial Data}}
\author{\small (in alphabetical order) Michael Dumelle and Matthew Higham and Others \thanks{Dumelle, Olsen, Kincaid, Weber}}
\date{\vspace{-2cm}}

% starting the document
\begin{document}

% printing the title (which is in the preamble)
\maketitle

% adding line numbers
\linenumbers

% page break for abstract
\clearpage

% inserting abstract
% \input{abstract.tex}

% page break for start of manuscript
\clearpage

%%%%%%%%%%%%%%%%%%%%%%%%%%%%%%%%%
% Overview
%%%%%%%%%%%%%%%%%%%%%%%%%%%%%%%%%
\section*{\centering {Overview}}

\subsection*{Why This Paper?}

We believe the distinction between these approaches is often misunderstood and there are several ways we could enchance the literature surrounding this topic:

\begin{enumerate}
  \item Spatial Design-Based vs Spatial Model-Based: There are no comparisons in the literature between spatial model-based approaches and *spatially balanced* design approaches. From what we have seen, these comparisons are between spatial model-based approaches and independent random sample designs. While important to study this behavior, this comparison is no longer fair nor modern. Spatially balanced sampling has exploded in popularity throughout the last decade, and the design-based vs model-based literature needs to reflect this trend.
  \item A ``fair'' comparison: We feel that literature in this area has considered scenarios that are more well-suited for the model-based scenario. For example, if you simulate a Gaussian erorr with an exponential covariance and then then compare design-based estimates to model-based estimates assuming an exponential covariance, of course the model-based approach will outperform design-based approaches. And these are the types of comparisons in the literature, which find model-based approaches generally yield more precise variance estiamtes. A challenge lies in creating a comparison scenario that is reasonable and intuitive. One this to consider would be exploring the comparison after estimating a misspecified covariance function using model-based approaches. 
  \item Finite AND infinite populations: Literature in the area focuses specifically on finite populations or infinite populations; we want to discuss both in detail.
  \item Pragmatic Focus: We see papers in this area tend to be fairly technical. We want the focus to be less on details, more on discussing the pragmatic questions practitioners will be faced with. For example, a thorough discussion of benefits and drawbacks of each method written for practicioners is warranted.
  \item Provide reliable software

\end{enumerate}

\subsection*{Initial Literature}

\begin{itemize}
  \item Design-Based Overview \citep{Sarndal2003model, Lohr2009sampling}
  \item Model-Based Overview \citep{Cressie2015statistics, Schabenberger2017statistical}
  \item Design-Based and Model-Based Comparisons \citep{Hansen1983evaluation, Brus1997random, VerHoef2002sampling, Cooper2006sampling, sterba2009alternative, Brus2020statistical, Chan2020bayesian}
  \item Spatially Balanced Design and Analysis \citep{StevensOlsen2003VarianceEstimation, StevensOlsen2004GRTS}
  \item Finite Population Block Kriging \citep{VerHoef2002sampling, VerHoef2008spatial, Higham2020adjusting}
\end{itemize}

\subsection*{Potential Journals}
\begin{itemize}
  \item Ecological Applications
  \item Methods in Ecology and Evolution
  \item Journal of Applied Ecology
  \item Environmetrics
  \item Environmental and Ecological Statistics
\end{itemize}

%%%%%%%%%%%%%%%%%%%%%%%%%%%%%%%%%
% Outline
%%%%%%%%%%%%%%%%%%%%%%%%%%%%%%%%%

\section*{\centering OUTLINE}

%%%%%%%%%%%%%%%%%%%%%%%%%%%%%%%%%
% Manuscript Start
%%%%%%%%%%%%%%%%%%%%%%%%%%%%%%%%%
\section{\centering INTRODUCTION}

%----------------------
% Background
%----------------------
\section{\centering BACKGROUND}


%----------------------
% Numerical Analysis
%----------------------
\section{\centering NUMERICAL ANALYSIS}

\subsection{Simulation-Based}

We would like to keep this section manageable. Perhaps we start with the following examples 
\begin{itemize}
  \item simulate via correct model (model outperforms sampling)
  \item simulate via slightly misspecified model (model still outperforms sampling)
  \item simulate via very misspecified model (sampling outperforms model)
  \item simulate via extremely misspecified model (e.g. counts with lots of zeroes and a lot of overdispersion) (neither does well)
\end{itemize}

\subsection{Data-Based}

\subsection{Software}



%----------------------
% Discussion
%----------------------
\section{\centering DISCUSSION}



%%%%%%%%%%%%%%%%%%%%%%%%%%%%%%%%%
% Document End
%%%%%%%%%%%%%%%%%%%%%%%%%%%%%%%%%


% page break for bibliography
\clearpage

% bibliography style
\bibliographystyle{apalike}

% bibliography
\bibliography{draft}

% ending the document
\end{document}
